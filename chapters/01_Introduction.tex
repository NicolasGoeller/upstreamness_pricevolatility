\section{Introduction}\
\label{sec:intro}

% What is the motivation
Within the last three years, the world economy has experienced its first episode of significant price fluctuations since the Global 
Financial Crisis in 2008. These price dynamics were induced first by the Covid crisis and then by the energy spill-overs from the 
Russia-Ukraine war. Research has shown that supply chain linkages have significantly impacted the increased volatility of prices. Evidence, 
in particular, suggests that the impact of an economic crisis is channeled from a few strongly affected sectors into the broader economy
through supply chain linkages. \textcite{labelle2022GlobalSupplyChain} document this for technology equipment and automobiles during 
the Covid crisis and \textcite{minton2023DelayedInflationSupply} do so for Petroleum Refining in case of the Russia-Ukraine war. In a 
third line of inquiry \textcite{amiti2021HighImportPrices} show that the prices of inputs imported to the US reacted strongly, in 
particular for unfinished industrial inputs during the Covid crisis (see figure (\ref{fig:covid_imports})). 

\begin{figure}[H]
    \centering
    \includegraphics[width=0.9\textwidth]{graphs/amiti_intro_plot_prices.png}
    \caption{\label{fig:covid_imports} Prices for Imported Goods in the US as taken from \textcite{amiti2021HighImportPrices}}
\end{figure}

Naturally, this event is unique in the size of its shocks and their global scale. With global supply chains spanning countries and 
sectors, price dynamics could be specific to a country, a sector in several countries, or a particular country-sector combination. Anecdotally,
however, for a coarse sectoral disaggregation, and for the US only, evidence as in figure (\ref{fig:covid_imports}) suggests
that sectors conventionally considered more upstream in the value chain are more volatile in their prices than downstream ones, especially since 
2021, but potentially even before. Such an observation appears to be in parallel to phenomena such as the ``Bullwhip effect'' commonly discussed in 
management science \parencite{sterman1989ModelingManagerialBehavior}. Consequently, this combination of anecdotal evidence and thematic commonality 
with a related discipline raises a general question about the relationship between the position of a country-sector in the 
global supply network and the volatility of its prices: Does the position of a country-sector along the value chain matter for its price 
volatility?

% why is this important?
The comovement and volatility of prices are covered by a significant body of literature that includes production systems with intermediate 
inputs from theoretical and empirical perspectives. Consequently, this paper is a theoretical exploration, specifically about the implication 
of a country-sector's position in the production chain toward price volatility. Within studies of Input-Output systems, such positions 
include the country-sector's distance to final demand and local input supply, respectively, and the diversification among its buyers and 
suppliers. My analysis's results should constitute a mapping from production chain position to price volatility and how various types of 
shocks affect this relationship.

In order to obtain a theoretically founded mapping of production chain positions to price volatility, I derive a model for N-sector cost 
minimization with Input-Output structures. This model simulates price outcomes for the upstream propagation of final demand shocks   
and the downstream propagation of primary input cost shocks. The proposed model and its numerically simulated results for both a 
1-Country, 2-Sector setup, and a 2-Country, 2-Sector setup reproduce the descriptive evidence derived through linear regression analysis
on empirical data for sectoral prices and IO linkages. The simulated model and empirical analysis show that as a country-sector's distance 
to final demand (upstreamness) increases, so does its price volatility. Similarly, as a country-sector's distance to primary inputs 
(downstreamness) increases, its price volatility should be observed to decrease. The first of these results is well in line with figure 
(\ref{fig:covid_imports}), where industrial supply prices change much stronger than consumer goods prices. The second result is harder to 
verify with this graph due to the difficulty in determining distances to primary inputs for the depicted sectors. In my model, both 
these relations hold even as shock transmission is simulated over several stages up (for final demand shocks) or down (for primary 
input cost shocks) the production chain. Further theoretical implications show that correlated shocks at the country or sector level 
do not break this general pattern but induce sectoral comovement as they decrease the differences in volatility for a simulated 
propagation of the shocks beyond a first-stage transmission.

% what am I doing?
The rest of the paper is structured as follows. Section \ref{sec:lit} provides an overview of the relevant literature, while section 
\ref{sec:methods}, \ref{sec:datafacts}, and \ref{sec:model} will discuss the methods, the stylized facts and the model and simulation 
framework for this paper, respectively. Section \ref{sec:simres} presents the results derived from the simulations under various 
Input-Output structures and their implications. Finally, section \ref{sec:limit} discusses the limitations of the approach utilized in 
this paper but also possible extensions for future research, and section \ref{sec:conc} concludes.


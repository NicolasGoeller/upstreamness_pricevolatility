\section{Limitations and Extensions}
\label{sec:limit}

The results presented in the previous section show that under the presented settings, the model is able to reproduce both the 
positive correlation between upstreamness and price volatility and the negative correlation between downstreamness and price volatility.
This should be seen as a success of the model since the model-based reproduction of these stylized facts was the primary goal 
of this paper. However, the model is still grossly oversimplified as a representation of a globally integrated production system and uses 
very coarse restrictions on implicit and explicit parameters. This section will outline some of the shortcomings that are visible already 
now, how the model might be modified to incorporate solutions to these shortcomings, and what the implications of such modifications would 
be. 

%% Oversight in setup change in openness implies wrong correlations between prod chain measures.
First, in setting up the IO matrices used for the simulations, I intentionally made it so that one matrix would have a negative correlation 
between upstreamness and downstreamness (or rather inverted rankings of these measures) and one would have a positive correlation between
these variables (parallel ranking). I also wanted to keep (country-)sectors at equal output size for the whole table in order to 
exclude the factor of size differences as a determinant of differences volatility and keep the range of parameters to explore at a 
sensible size. The resulting IO matrices had the correlations as intended, but the one that showed a ``high openness'' has a negative 
correlation between upstreamness and downstreamness, making this feature explicitly stand in contrast to the data. While this is 
a problematic feature of the setup, I decided it was the preferable option compared to opening the project up to a discussion 
about sector size since that would have extended into issues about granularity. Including the issue of granular sectoral economies would 
have exceeded the scope of this thesis in terms of model formulation and the availability of disaggregated data. However, this is 
certainly an extension to the project that is worthwhile to consider for the future.


%% Issues with structural parameters
Second, the model presented in section \ref{sec:model} has no inter-industry barriers or frictions. Most notably, 
such elements include trade costs and pass-through. Regarding trade costs, such a modification would be easily achievable, e.g. in the 
form of standard ``iceberg'' trade costs as a square matrix of constants with values larger or equal one to represent either the price increase 
for any inter-industry purchase. This addition might be instrumental in making simulation scenarios with multiple countries more 
realistic. The chosen values should be systematically varied to explore the sensitivity of results to different trade costs or 
undergo empirical justification (for a large-scale quantitative model implementation). Regarding pass-through, there is a breadth of empirical 
studies to show that pass-through is often imperfect but may also be heterogeneous across sectors (or firms) and dependent on the state of the 
economy (e.g. inflation rate). A simple modification to incorporate pass-through as a vector of constants applied to the marginal cost 
equation would not be sufficient since these constants would cancel each other out when reformulating the model into price changes.
However, a more serious attempt at including heterogeneous pass-through would have to decide whether to include such a feature as 
price rigidity or variable mark-ups and hence rely on models of price setting or imperfect competition to supplement the current framework.
Such an extension would also make the treatment of non-constant Returns-To-Scale values, as discussed in section \ref{sec:model}, significantly
more coherent. The inclusion of models of price setting into the model framework would have the additional advantage of modeling a realistic
distribution of price changes, especially in terms of frequency e.g. via menu costs models. The current framework has no such inhibition,
which should imply price changes that are too frequent and, hence, too high volatility. However, such a model would necessarily 
have to include a dynamic version of the optimization problem since fixing prices over time requires such a choice to be dynamically optimal 
(at least in expectation).

Third, the model framework I use to generate my results relies on a strong assumption of its Input-Output structure being exogenous.
However, firms may account for the expected price volatility of their supplier when making a sourcing decision.
Such an endogeneity in the data-generating process behind my model framework would not make the observation of the effects of
small, idiosyncratic shocks on price volatility infeasible in and of itself. However, it does introduce another potential explanation 
for the observed relations between production chain position and price volatility. With the formation of IO links being endogenous on 
the expected price volatility, it could be that sectors with naturally high price volatility are placed further upstream as a result 
of this process.

%% Issues with shock specification
Fourth, it should be an aspect of some criticism that the simulations are based on relatively arbitrary shock distributions. The choice 
of a normal distribution with mean zero for shocks to final demand, primary input cost, and initial price changes is not a problem as 
such since it is a convenient and tractable way to introduce variation of stochastic components to the model. The first moment of the 
empirically observed cross-sectional price change distribution is slightly larger than zero since inflation is the usual trend (rather 
than deflation). Hence, while a first moment of zero is uncritical for the shock distributions, it evidently is inconsistent with 
empirical facts on price change distributions. However, this paper aims to explain simple patterns observed in the data through a basic 
model framework and not match volatilities quantitatively; such inconsistency is still acceptable. The justification for the variances 
of these distributions is to be seen more critically since a key part of the simulation result is a visual judgment of the relative 
importance of shocks and initial price changes. This limitation could be fixed somewhat through a more fine-grained exploration of 
different variance values and trying to find the parameter combinations that correspond to decent fits with empirically observed price 
volatilities. A more rigorous attempt would involve a full model calibration of the distribution parameters to match empirical stylized 
facts, e.g. moments of the price change distributions.

%% Issues with data and external validity
%Fifth, lack of data on prices for service sectors restricts the analysis in this paper significantly. Obviously, this is a common 
%problem for studies with sectoral disaggregation since data on sectors is much more difficult to survey and harmonise across countries. 
%However, the exclusion of service sectors in the case of this model systematically increases the share of
%Does not include data on services - pushes them into the value added component, which is a gross simplification since services
%might still be traded, especially within a country





\section{Multisector Model with Intermediate Inputs}
\label{sec:model}

After the basic concepts for this paper have been defined in the previous sections, I will now outline a basic model framework of cost 
minimization. I utilize this model, heavily inspired by \textcite{caliendo2015EstimatesTradeWelfare}, to provide an intuition to explain 
the effect of shocks on final demand and cost of primary inputs under a partial equilibrium in the goods market. Simulations of upstream 
and downstream propagation can be conducted with this framework for various Input-Output structures and shock scenarios. This should 
generate an understanding of the relationship between the position of a sector in the value chain and its price volatility. The previous 
explanation of upstreamness and downstreamness, this model outline, and the specification of toy Input-Output tables will all follow the 
same notational conventions regarding intermediate inputs, input coefficients, total output, and final demand. 

\subsection{Model Outline for Cost Minimization}

With an N-sector economy, the corresponding optimization problem of a representative firm in industry $i$ contains $N+1$ factor inputs 
(N intermediates $\{z_{ji}\}_{j=1}^N $, one local primary input $l_i$, e.g. labor, services) and their prices 
$w, \hspace{3pt}\{p_{j}\}_{j=1}^N$. As such, this implies the existence of N+1 markets for factor inputs, all of which are assumed to 
be perfectly competitive for this paper. For simplicity, a Cobb-Douglas production function is used. This functional form is a strong 
assumption since it imposes constant shares for all factors of production, hence an exogenous production structure. However, it also 
implies an elasticity of substitution between the production factors equal to one, which implies strong substitutability of inputs. 
This is a significant deviation from the Leontief production functions, usually employed for Input-Output models. Hence, the model 
constructed here might be expected to yield propagation results that are more conservative than a similar Leontief-type model.
Finally, a Cobb-Douglas production also implies that optimal production will have to employ inputs at equal orders of magnitude for 
all inputs with non-zero input shares. This assumption is increasingly unrealistic as the number of factors rises. I refer to the 
factors shares for this production function as input shares for the primary $\alpha_{0i}$  and intermediate $\{\alpha_{ji}\}_{j=1}^N$ 
inputs. The matrix $A$ of these intermediate input shares is equivalent to the matrix $A$ of input coefficients in section 
\ref{sec:methods} and hence constitutes the Input-Output structure for this model. I also include a labor productivity variable 
$\tau_i$ that applies to primary inputs. All variables are denoted with a singular sector index, e.g. $i$, but this can be 
easily extended for country-sector indices. I write the corresponding minimization problem for cost $C_i$ as follows:

\begin{equation} \label{eq:min_prob}
\begin{split}
    &\min_{\substack{l_i, \{m_{ji}\}_{j=1}^N}} \mspace{5mu} C_i = wl_i + \sum_{j=1}^N p_j z_{ji} \\
   s.t. \mspace{10mu} &x_i = (\tau_i l_i)^{\alpha_{0i}} \prod_{j=1}^N z_{ji}^{\alpha_{ji}}
\end{split}
\end{equation}

For notational simplicity, in the following I will abbreviate $Z_i = \prod_{j=1}^N z_{ji}^{\alpha_{ji}}$. Furthermore, it must be noted
that in most of the literature, the sum of input shares $\sum_{j=0}^N \alpha_{ji} = \gamma_i$ is assumed to be equal to 1, implying 
constant returns-to-scale. However, this project will explicitly not make this assumption due to the short-term interest of the 
question asked and hence explore the impact of different returns-to-scale (as compared to classical IO models mentioned in section 
\ref{sec:methods}). Consequently, one can write the N+1 FOCs of the above optimisation problem, where $\mu_i$ is the Lagrange multiplier:

\begin{equation} \label{eq:FOC}
\begin{split}
    & w - \mu_i\alpha_{0i}\tau_i^{\alpha_{0i}}l_i^{\alpha_{0i} -1} Z_i =0\\
    & p_j - \mu_i(\tau_i l_i)^{\alpha_{0i}} \alpha_{ji} z_{ji}^{-1} Z_i = 0 \hspace{10pt} \forall j
\end{split}
\end{equation}

This, in turn, allows me to derive the MRTS of factor inputs and their conditional factor demands from substituting into the
production function.

\begin{equation} \label{eq:CFD}
\begin{split}
    & l_i =  x_i^{\frac{1}{\gamma_i}} \left(\frac{\alpha_{0i}}{w \tau_i} \right)^{\frac{\gamma_i -\alpha_{0i}}{\gamma_i}} \prod_{j=1}^N \left(\frac{p_j}{\alpha_{ji}}\right)^{\frac{\alpha_{ji}}{\gamma_i}} \\
    & z_{ji} =  x_i^{\frac{1}{\gamma_i}} \left(\frac{w}{\alpha_{0i}\tau_i}\right)^{\frac{\alpha_{0i}}{\gamma_i}} \frac{\alpha_{ji}}{p_j} \prod_{k=1}^N \left( \frac{p_k}{\alpha_{ki}} \right)^{\frac{\alpha_{ki}}{\gamma_i}}
\end{split}
\end{equation}

By inserting these factor demands into the cost function, I obtain its final formulation and can derive by output $x_i$ for the marginal
cost function with marginal cost $mc_i$ on the LHS. This function is written with the variables demand $x_i$, primary input unit cost 
$\frac{w}{\tau_i}$, intermediate input prices $\{p_j\}_{j=1}^N$, and has the input shares $\{\alpha_{ji}\}_{j=0}^N$ and returns-to-scale 
indicator $\gamma_i$ as parameters. Going forward, I also denote the primary input unit cost as $\omega_i = \frac{w}{\tau_i}$.
\footnote{
The term $\Theta_i$ denotes a firm-specific (sector-specific since I assume representative firms) constant. Specifically, 
$\Theta_i = \prod_{j=0}^N \left(\frac{1}{\alpha_{ji}}\right)^{\frac{\alpha_{ji}}{\gamma_i}}$ This notation can be found in a similar 
manner with \textcite[footnote 19]{caliendo2015EstimatesTradeWelfare}.}

\begin{equation} \label{eq:cost_func}
\begin{split}
    C_i &= w l_i + \sum_{j=1}^N p_j z_{ji} \\
    C_i &= \gamma_i x_i^{\frac{1}{\gamma_i}} \left(\omega_i\right)^{\frac{\alpha_{0i}}{\gamma_i}} \prod_{j=1}^N p_j^{\frac{\alpha_{ji}}{\gamma_i}} \Theta_i\\
    mc_i =\frac{\partial C_i}{\partial x_i} &= x_i^{\frac{1 - \gamma_i}{\gamma_i}} \left(\omega_i \right)^{\frac{\alpha_{0i}}{\gamma_i}} \prod_{j=1}^N p_j^{\frac{\alpha_{ji}}{\gamma_i}} \Theta_i
\end{split}
\end{equation}

In order to observe variations in this price-setting mechanism, I introduce the potential for shocks to final demand and labor cost. 
The demand shock requires output to be decomposed into a sum final demand and intermediates sales $x_i = \sum_{j=1}^N z_{ij} + f_i = S_i + f_i$. 
This decomposition is a relevant feature of the model since the theoretical framework has a strong connection to the share of final demand 
in total output $\lambda_i$ that should also be reflected in the model's structure. For simplicity, it is assumed that the shocks are drawn 
with a mean zero from a normal distribution and apply to an exogenous constant, which is the final demand component and the effective wage 
in this framework. Thus, this way of constructing shocks will produce random fluctuations of the shocked variable around its partial 
equilibrium state. I will consider shocks to final demand and to the local primary input's unit cost. The shock to the primary input's 
unit cost is defined in a log-normal manner for equation (\ref{eq:cost_shock}), as is standard in the literature. As for the shock to final
demand, the additive decomposition of demand necessitates that I adopt a slightly different approach, seen in 
equation (\ref{eq:demand_shock}), since I need to take the logs of this demand component at a later stage in equation 
(\ref{eq:demand_change}).\footnote{Numerically, there is a difference between this way of writing the demand shock and doing so in the more 
standard, log-normal fashion. However, the difference is quite small, and the analytical layout remains reasonably easy to interpret. Thus, 
this approach was deemed acceptable, given the benefits for the model shock structure at later stages.} Technically, this demand shock could 
make the final demand negative ($\eta_i < -1$), leading to theoretically implausible values. I exclude this possibility by setting a logical 
condition of minimal shock values in my code. Hence, I write as follows:

\begin{equation} \label{eq:demand_shock}
    x_i = S_i + f_i (1+\eta_i),  \hspace{10pt} \eta_i \sim \mathcal{N}(0,\sigma_{\eta})
    %x_i = S_i + f_i e^{\eta_i},  \hspace{10pt} \eta_i \sim \mathcal{N}(0,\sigma_{\eta})
\end{equation}
\begin{equation} \label{eq:cost_shock}
    \omega_i =  \frac{w}{\tau_i} e^{\varepsilon_i}, \hspace{10pt} \varepsilon_i \sim \mathcal{N}(0,\sigma_{\varepsilon})
\end{equation}

With this, the final marginal cost function with shocks writes:
\begin{equation}\label{eq:mc_cost}
    p_i = mc_i = \left( S_i + f_i (1+\eta_i) \right)^{\frac{1- \gamma_i}{\gamma_i}} \left(\frac{w}{\tau_i} e^{\varepsilon_i} \right)^{\frac{\alpha_{0i}}{\gamma_i}} \prod_{j=1}^N p_j^{\frac{\alpha_{ji}}{\gamma_i}} \Theta_i
    %mc_i = \left( f_i e^{\eta_i} + S_i \right)^{\frac{1- \gamma_i}{\gamma_i}} \left(\frac{w}{\tau_i} e^{\varepsilon_i} \right)^{\frac{\alpha_{0i}}{\gamma_i}} \prod_{j=1}^N p_j^{\frac{\alpha_{ji}}{\gamma_i}} \prod_{j=0}^N \left(\frac{1}{\alpha_{ji}}\right)^{\frac{\alpha_{ji}}{\gamma_i}}
\end{equation}

Under the assumptions that (1) price setting is simultaneous across sectors, (2) inputs are sourced and used at prices set in the 
previous period, (3) the input-output structure is exogenous, (4) pass-through is complete (e.g. lack of price rigidities, etc.),  
(5) competition is perfect, and (6) each sector consists of one representative firm, it is possible to derive simulated, sectoral price 
series from equation (\ref{eq:mc_cost}). Among these assumptions, (1) is trivial since firms do not wait for other firms from other 
sectors to set their price. Assumption (2) has to be understood in the context of the partial equilibrium model framework, where 
deviations of sectoral prices can exist at different stages of the production chain. Finally, assumptions (3), (4), (5), and (6) are 
acceptable at this point because they make working with this framework tractable without impeding the clarity of the mechanism to be 
shown here. Especially, assumption (5) on perfect competition is to be seen critically here since I intentionally do not assume 
constant Returns-To-Scale. However, under non-constant Returns-To-Scale, it is unlikely for competition to be strong enough to push 
profits to zero and hence validate this assumption. Given that the non-shock terms of equation (\ref{eq:mc_cost}) are derived as a result
of optimization under partial equilibrium, it may possible to see these shock responses as stochastic fluctuations around an optimal state that 
balance out in the time series average. Nonetheless, this is clearly imperfect, ignoring the potential for profits and especially 
profit-seeking firm behavior. I further discuss how one could relax this and the other assumptions in section \ref{sec:limit} and what 
this might imply for the results. 


In applying the equation (\ref{eq:mc_cost}) on itself, each vector of output prices is the vector of input prices in the next 
iteration. However, this marginal cost equation is extremely sensitive to the magnitude of price levels, making the overall simulation 
process very dependent on the ``equilibrium prices''. Such prices will be derived by any system of production linkages (and associated 
primary input costs, demands, productivities, etc.) after several iterations without any exogenous shocks (draws of 
$\eta_i, \varepsilon_i$) at reasonable accuracy. The sensitivity of price setting to the level of input prices reflects the well-known 
empirical fact about differences between price distributions across different countries and sectors (primarily their first moment). 
Since the distributions of price changes are much more comparable across countries and sectors, a model of the theoretical problem 
above but written in price changes might be more appropriate to represent the intended phenomenon. Regarding the formulation of price 
changes in equation (\ref{eq:deltap}), the first step towards this new model is to write the marginal cost in logs and with a time 
subscript (while taking out the shock decomposition for notational simplicity).

%Since volatility is defined as the variance of the relative price changes observed over a particular period, the model has to produce a 
%simulated output in form of a time series prices to compute price changes. If I can make the assumption that price setting is 
%simultaneous across sectors, but inputs are used at prices produced in the previous time period, then the input prices and the prices
%implied by the marginal cost will potentially be different. Then it is possible, to obtain a time series of simulated prices, by 
%iterating equation (\ref{eq:mc_cost}) multiple times, making each output of prices the input prices in the next iteration. If the structure
%of the production network remains unchanged over the simulation (assumed exogenous), any differences in prices should be due to shocks
%to the sector, its direct upstream suppliers and the higher-order upstream suppliers in the previous period(s).
%It is important to note 
%here that this equation is extremely sensitive to the magnitude of price level, making the overall simulation process very dependent on 
%the ``equilibrium''. Any system of production linkages (and associated wages, demands, productivities, etc.) will yield a vector of 
%``equilibrium'' that this function will mechanically arrive at after several iterations without any exogenous shocks ($\eta_i, 
%\varepsilon_i$). This dependence of the level-based simulation process corresponds to the large differences between empirically observed price 
%distributions across different countries and sectors (primarily their first moment). However, the distributions of price changes are 
%much more comparable across countries and sectors. A model of the theoretical problem above but written in price changes might hence be 
%more appropriate to represent the intended phenomenon. In reference to the formulation of price changes in (\ref{eq:deltap}), the first 
%step towards this new model is to write the marginal cost in logs and with a time subscript (while taking out the shock decomposition 
%for notational simplicity).

%It is important to note here that this equation is extremely sensitive to the magnitude of price level, making the overall simulation 
%process very dependent on the ``equilibrium''. Any system of production linkages (and associated wages, demands, productivities, etc.)
%will yield a vector of ``equilibrium'' that this function will mechanically arrive at after several iterations without any 
%exogenous shocks. If the starting price is sufficiently far away from equilibrium, the new price will be influenced moreso by the move 
%towards ``equilibrium'' than the response to shocks. For this reason, I allow for several iterations of this cost function until all 
%price changes of the iteration are below a ``convergence threshold''. Since I am  strictly interested in the accumulation/ dispersion of 
%shocks through a system of production linkages, it is necessary to start introducing shocks with a vector of initial prices at or near 
%this ``equilibrium''. 


\begin{equation} \label{eq:log_mc}
    ln(p_{i,t}) = ln(mc_{i,t}) = \frac{1-\gamma_i}{\gamma_i} ln(x_{i,t}) + \frac{\alpha_{0i}}{\gamma_i} ln(\omega_{i,t}) + \sum_{j=1}^N \frac{\alpha_{ji}}{\gamma_i} ln(p_{i,t-1}) + ln(\Theta_i)
\end{equation}

Recalling equation (\ref{eq:mc_cost}), I can write the relative price change $\Delta p_{i,t+1}$ with this log marginal cost:

\begin{equation}
    \begin{split}
        \Delta p_{i,t+1} &= ln(p_{i,t+1}) - ln(p_{i,t}) \\
        &= \frac{1-\gamma_i}{\gamma_i} [ln(x_{i,t+1}) - ln(x_{i,t})] + \frac{\alpha_{0i}}{\gamma_i} [ln(\omega_{i,t+1}) - ln(\omega_{i,t})] + \\ 
        & \sum_{j=1}^N \frac{\alpha_{ji}}{\gamma_i} [ln(p_{i,t}) - ln(p_{i,t-1})] \\
        &= \frac{1-\gamma_i}{\gamma_i} \Delta x_{i,t+1} + \frac{\alpha_{0i}}{\gamma_i} \Delta \omega_{i,t+1} + \sum_{j=1}^N \frac{\alpha_{ji}}{\gamma_i} \Delta p_{i,t} 
    \end{split}
\end{equation}

In order to complete the reformulation as a price change equation, I now re-consider the shocks to demand (\ref{eq:demand_shock}) and 
primary input cost (\ref{eq:cost_shock}), and rewrite them as relative changes. Since $\omega_i$ has a simple multiplicative structure, the 
changes result trivially in the absolute change of the shock draws, denoted by $d \varepsilon_{i,t+1}$. However, in the case of reformulating 
the demand shock, intermediate sales as a constant necessitate more extensive steps. After normalizing the demand on the sum of intermediate 
sales and constant final demand, the remaining term of $ln(1+x)$ can be written as an infinite geometric sum through the Taylor expansion. 
The last step holds as a first-order approximation of the actual value at the first order. Given that the shocks are drawn from the same stationary 
distribution $\eta_i \sim \mathcal{N}(0,\sigma_{\eta})$ at relatively small variance values, the likelihood of cases to occur where this
approximation fails grossly is considered small. Hence, it was deemed appropriate. For simplicity of notation, I write the share of final 
demand in output as $\frac{f_i}{(S_i+f_i)} = \lambda_i$. The results can now be written as the product of final demand share and the absolute 
change in final demand shock $\lambda_i d \eta_{i,t+1}$. This would not have been possible under a log-normal shock formulation, thus justifying 
the deviation from the standard in the literature. 

\begin{equation} \label{eq:cost_change}
    \begin{split}
        \Delta \omega_{i,t+1} &= ln(\omega_{i,t+1}) - ln(\omega_{i,t}) \\
        &= ln(\frac{w}{\tau_i} e^{\varepsilon_{i,t+1}}) - ln(\frac{w}{\tau_i} e^{\varepsilon_{i,t}}) \\
        &= ln(\frac{w}{\tau_i}) + ln(e^{\varepsilon_{i,t+1}}) - ln(\frac{w}{\tau_i}) - ln(e^{\varepsilon_{i,t}}) \\
        &= \varepsilon_{i,t+1} - \varepsilon_{i,t} \\
        &= d \varepsilon_{i,t+1}
    \end{split}
\end{equation}

\begin{equation} \label{eq:demand_change}
    \begin{split}
        \Delta x_{i,t+1} &= ln(x_{i,t+1}) - ln(x_{i,t}) \\
        &= ln(S_i + f_i(1+\eta_{i,t+1})) - ln(S_i + f_i(1+\eta_{i,t}))\\
        &= ln\left((1 + \frac{f_i}{S_i + f_i}\eta_{i,t+1}) (S_i+f_i) \right) - ln\left((1 + \frac{f_i}{(S_i+f_i)} \eta_{i,t}) (S_i+f_i) \right) \\
        &= ln(1 + \frac{f_i}{(S_i+f_i)} \eta_{i,t+1}) + ln((S_i+f_i)) - ln(1 + \frac{f_i}{(S_i+f_i)} \eta_{i,t}) - ln((S_i+f_i)) \\
        &= ln(1 + \frac{f_i}{(S_i+f_i)} \eta_{i,t+1}) - ln(1 + \frac{f_i}{(S_i+f_i)} \eta_{i,t}) \\
        &= \sum_{n=1}^{\infty} \frac{(-1)^{n+1}}{n} \left(\frac{f_i}{(S_i+f_i)} \eta_{i,t+1}\right)^n - \sum_{n=1}^{\infty} \frac{(-1)^{n+1}}{n} \left(\frac{f_i}{(S_i+f_i)} \eta_{i,t}\right)^n \\
        &= \sum_{n=1}^{\infty} \frac{(-1)^{n+1}}{n} \left(\frac{f_i}{(S_i+f_i)}\right)^n (\eta_{i,t+1}^n - \eta_{i,t}^n) \\
        &= \frac{f_i}{(S_i+f_i)} d \eta_{i,t+1} \\
        &= \lambda_i d \eta_{i,t+1}
    \end{split}
\end{equation}

With the reformulation of these two shocks, I can now write the final version of the price changes model. Price changes in a period can 
be attributed to either a change in intermediate input prices, a change in final demand, or a change in primary input cost. 
Furthermore, this equation is now stationary and does not rely on the strict temporal dimension to implement a simulation model. 
Hence, I write intermediate input price changes as supplier price changes $\Delta p_{i,s}$ and the output price changes as buyer price 
changes $\Delta p_{i,b}$.
\begin{equation} \label{eq:price_changes}
    \Delta p_{i,b} = \frac{1-\gamma_i}{\gamma_i}  \lambda_i d \eta_{i} + \frac{\alpha_{0i}}{\gamma_i} d \varepsilon_{i} + \sum_{j=1}^N \frac{\alpha_{ji}}{\gamma_i} \Delta p_{i,s} \\
\end{equation}

\subsection{Simulation Procedure}

From equation (\ref{eq:price_changes}), simultaneous draws of initial supplier price changes $\Delta P_s$, and the two exogenous shocks 
(final demand and primary input cost) will yield hypothetical price changes, providing a stationary representation of PPI inflation. The 
volatility of sectoral prices over this set of price changes is the first simulated outcome to consider. While equation 
(\ref{eq:price_changes}) dictates the formal structure for this process, it is convenient to rewrite this equation in matrix form. For 
this purpose, I denote the vectors of price changes for buyers and suppliers $\Delta P_{b}, \Delta P_{s}$, vectors of shocks to final 
demand and input cost $d \mathbb{H}, d \mathbb{E}$, the matrix of technical coefficients $\mathbb{A}$ and the diagonal matrices of 
primary input shares, final demand shares and return-to-scale indicators $\hat{\mathbb{A}}_0, \hat{\Lambda}, \hat{\Gamma}$. This 
equation can be understood as the downstream transmission of shocks to final demand and primary input cost together with an initial 
change of intermediate input prices. These shocks are moderated in their impact by their shares of primary input, final demand, and 
intermediate inputs, as these shares represent exposure to shocks from the respective sources. The Returns-To-Scale indicator will, on 
its own, either amplify (for DRS) or diminish (for IRS) the shock impacts. Such an understanding is consistent with the micro-founded 
perspective of cost-minimizing firms that change their prices corresponding to changes in their marginal costs caused by changes to the 
cost of primary and intermediate inputs and final demand.

\begin{equation} \label{eq:downstream_changes}
    \Delta P_{b} = (\hat{\Gamma}^{-1} - I) \hat{\Lambda} d \mathbb{H} + \hat{\Gamma}^{-1} \hat{\mathbb{A}}_0 d \mathbb{E} + \Delta P_{s} \mathbb{A} \hat{\Gamma}^{-1}
\end{equation}

In order to obtain an equation that describes the transmission of price changes and shocks in an upstream direction, similar as in equation 
(\ref{eq:downstream_changes}) does for downstream transmission, one could simply invert the equation to isolate $\Delta P_s$. This 
equation, however, contains the matrix $A^{-1}$, which does not have a sensible interpretation for input-output models, but it also tends 
to hold values that lead to unrealistically large price responses on the given shock input. Hence, I decided against simply choosing 
shock variance values that would return price changes of reasonable size and instead opted to impose an ad-hoc modification of the 
equation, using the matrix of output coefficients $\mathbb{B}$, which is defined as $\mathbb{B} = \hat{X}^{-1} Z$ (in parallel to 
$\mathbb{A}$ being defined as $\mathbb{A} = Z \hat{X}^{-1}$).  Although the equation is not micro-founded as a result of firm 
cost minimization, it can be interpreted similarly to the equation for downstream transmission. The upstream pass-through of price 
pressures can be understood in two steps, where first, the initial change of output prices is modified by the shocks to final demand or 
primary input costs. Since pass-through must be complete in this equation, the remaining change of prices will be transmitted through 
the matrix $\mathbb{B}$, which indicates how much an industry relies on each of its buyers in terms of sales. If all or a sales-weighted 
majority of buyers report a net negative price change (after shocks to final demand and primary input cost), then the industry must 
accommodate this price movement.

\begin{equation} \label{eq:upstream_changes}
    \Delta P_{s} = \hat{\Gamma}^{-1} \mathbb{B} [\Delta P_{b} - (\hat{\Gamma}^{-1} - I) \hat{\Lambda} d \mathbb{H} - \hat{\Gamma}^{-1} \hat{\mathbb{A}}_0 d \mathbb{E}]
\end{equation}

Both equations on the transmission of price changes presented just now describe this process for a singular stage of the production chain.
However, since the model presented here has a round-about structure of intermediate input sourcing, it is possible to consider price 
changes N stages down or up from an initial price draw by applying the function to itself N times over. In doing so, the simulation tries
to disentangle the propagation of shocks along the time dimension, where exogenous shocks would change as the propagation occurs, from the
propagation along the value chain dimension, where exogenous shocks are still present but the price of inputs or outputs has already been 
affected. Consequently, shocks will accumulate as they are re-introduced with every new stage, making this kind of simulation only a 
realistic scenario for a small number of stages since firms would otherwise try to adapt their input mix to balance out the price change. 
Such a situation of restricted adaptation is exactly to consider under very rapid shock occurrence scenarios or with disrupted supply 
chains (see the paper by \textcite{labelle2022GlobalSupplyChain}). Furthermore, with this explicit consideration of the transmission of 
shocks over several stages of the production chains, it seems sensible to introduce the common assumptions that (1) cost shocks are 
transmitted downstream, and (2) demand shocks are transmitted upstream.\footnote{These assumptions are supported by theoretical and 
empirical findings from \textcite{acemoglu2016NetworksMacroeconomyEmpirical}} Such a restriction is sensible for the dimensionality of 
the simulation parameters to consider. Furthermore, it is reasonable not to impose both shocks at all stages of propagation since each 
stage represents an upstream or downstream movement on the production chain, implying a move away from the incidence of final demand 
and primary input cost, respectively. The concrete equations to compute this propagated price changes over several downstream and 
upstream stages are as follows:

\begin{equation} \label{eq:propagation}
\begin{split}
    \Delta P_b^n &= \Delta P_s (\mathbb{A} \hat{\Gamma}^{-1})^n  + \sum_{k=0}^{n-1} (\mathbb{A} \hat{\Gamma}^{-1})^k \hat{\mathbb{A}}_0 d \mathbb{E} \\
    \Delta P_s^n &= (\hat{\Gamma}^{-1} \mathbb{B})^n \Delta P_b - \sum_{k=0}^{n-1} (\hat{\Gamma}^{-1} \mathbb{B})^k (\hat{\Gamma}^{-1} - I) \hat{\Lambda} d \mathbb{H}
\end{split}
\end{equation}

With the equations and intuition for the simulation outlined as such, I can distinguish the inputs required to conduct an investigation
with them into structural ones (e.g. production structure, shares of final demand, and primary input) and variable ones (e.g. shock 
variance, variance of initial price change, Returns-To-Scale). I will vary the latter systematically to explore the sensitivity of 
results to these input choices, while I will choose the Input-Output structures deliberately to draw out certain behaviors through the 
simulation.

\subsection{1-Country, 2-Sector Setup}

%Intent of simple model setup (or two)

A  1-Country, 2-Sector economy has all the essential features to begin the exploration of the relationship between shock exposures 
implied by upstreamness and downstreamness and how that drives differences in price volatility. Even this simplified production structure 
can be used to model ``round-about'' Input-Output linkages while also showing differences in the shares of final demand and primary input 
use relative to total output. Naturally, such a setup is unrealistic, but it is sufficient to draw out the implications of sectoral 
interdependences and differences in shock exposure for price volatility. Any comparison of production chain position and price volatility
will have to be ordinal in this context since, with such a simple framework, ratios of these measures hold little value for interpretation.
With this in mind, I decided on two intentionally coarse IO matrices, where the first matrix (\ref{fig:IO1x2_lowopen}) has its values for 
final demand shares and primary input shares ranked the same over sectors, while the second matrix (\ref{fig:IO1x2_highopen}) has ranked 
them in an inverted way. For both matrices, the columns titled $X$ and $F$ represent total output and final demand, while the row with the
title $VA$ holds values for value added, previously written as $\omega_i$. Finally, the matrix with row and column names (s1,s2) has the 
values for purchases and sales of intermediate inputs, previously written as $Z$. Since I wanted to keep both sectors at the same size 
of output, this setup of IO linkages has led to matrix (\ref{fig:IO1x2_lowopen}) having a dominant diagonal (for rows and columns), 
while the off-diagonal is dominant for matrix (\ref{fig:IO1x2_highopen}) (for rows and columns). In terms of generalization for larger 
IO matrices, this implies that producers in the system represented by matrix (\ref{fig:IO1x2_lowopen}) rely majorly on ``domestic'' 
inputs, while producers in the system for matrix (\ref{fig:IO1x2_highopen}) do so with ``foreign'' inputs. I term these IO linkage 
scenarios ``low openness'' and ``high openness'' respectively.

\begin{figure}[H]
    \makebox[\textwidth][c]{%
    \subfloat[IO matrix for 1-Country, 2-Sector setup with low openness]{\scalebox{0.8}{\includegraphics{pictures/IO_table_1x2_lowopen.png}}
    \label{fig:IO1x2_lowopen}}
    \quad
    \subfloat[IO matrix for  1-Country, 2-Sector setup with high openness]{
        \scalebox{0.8}{\includegraphics{pictures/IO_table_1x2_highopen.png}}
        \label{fig:IO1x2_highopen}}
    }
\end{figure}

From the IO matrix (\ref{fig:IO1x2_lowopen}), it is apparent that sector $s1$ has a final demand share of 0.2 and a primary input share of 0.2.
Conversely, sector $s2$ has a final demand share of 0.4 and a primary input share of 0.4. This IO structure results in upstreamness values of
$s1: 4.286,\hspace{3mm} s2: 2.857$ and downstreamness values of $s1: 4.286,\hspace{3mm} s2: 2.857$. 
Evidently, with the symmetric structure of this IO matrix, the upstreamness and downstreamness values here are the same. Further, this structure of 
low openness (dominant matrix diagonal) has also created a positive ``correlation'', between upstreamness and downstreamness.
From the IO matrix (\ref{fig:IO1x2_highopen}), it is apparent that sector $s1$ has a final demand share of 0.3 and a primary input share of 0.5.
Conversely, sector $s2$ has a final demand share of 0.5 and a primary input share of 0.3. This IO structure results in upstreamness values of
$s1: 3.478,\hspace{3mm} s2: 3.043$ and downstreamness values of $s1: 3.043,\hspace{3mm} s2: 3.478$.
Evidently, with the symmetric structure of this IO matrix, the upstreamness and downstreamness values here are the same. Further, this structure of 
high openness (dominant matrix diagonal) has also created a negative ``correlation'', between upstreamness and downstreamness.

\subsection{2-Country, 2-Sector Setup}

Given my theoretical framework, a 1-Country, 2-Sector economy already has all the essential features to explore the basic relationship 
between production chain position and price volatility. Based on that, the 2-Country, 2-Sector setup presents a first step toward a more 
complete representation of a globally integrated production system. This applies to several specific features, specifically increased 
diversification of sourcing and sales across the various country-sectors, but also shocks to final demand and primary input cost that 
are correlated by sector or country next to basic country-sector shocks (regimes of mixed origin shocks are also possible). Such an IO 
matrix with two countries and two sectors each is not a realistic representation of a globally integrated production system, but it does 
strike a reasonable balance between the tractability of a simple model and the dangers of building a ``black box'' model in trying to 
gain external validity. This has been critiqued by \textcite{caliendo2015EstimatesTradeWelfare} about previous CGE-based approaches, 
and a full extension of my framework to the full extent of available data would incur the same problem. Finally, the assembly of four 
country-sectors with this setup will be instrumental in ensuring that any conclusions about rankings of volatility do not arrive because 
the sector with lower price volatility has a certain position on the production chain coincidentally. With these things in mind, I construe
two IO matrices where, again, for the first matrix (\ref{fig:IO2x2_lowopen}) the rankings of final demand share and value added share 
align, and where for the second matrix (\ref{fig:IO2x2_highopen}) these rankings are inverted. Overall, I have decided to introduce 
the shares for primary inputs and final demand so that the sector-wise comparison is clear, as is the country-wise comparison. 
However, a general structure in each country remains, with one sector more dependent on final demand or primary inputs respectively.
The notations of the matrices depicted 
are the same as for the 1-Country, 2-Sector with output $X$, final demand $F$, value added $VA$ and intermediate inputs $Z$. The only 
difference is that the country-sector denominations now include a country indexation (A\_s1,A\_s2,B\_s1,B\_s2). With all country-sectors 
of the same size in terms of output, this creates scenarios with matrix in figure (\ref{fig:IO2x2_lowopen}) having a dominant block-diagonal,
and the matrix in figure (\ref{fig:IO2x2_highopen}) having a dominant off-diagonal blockwise. Similarly to before, a dominant block-diagonal
for rows and columns implies that country-sectors rely on input sourcing and output sales on domestic partners at a majority (weakly).

\begin{figure}[H]
    \centering 
    \includegraphics[width=7.5cm]{pictures/IO_table_2x2_lowopen.png}
    \caption{\label{fig:IO2x2_lowopen} IO matrix for 2-Country, 2-Sector setup with low openness}
    
\end{figure}

\begin{figure}[H]
    \centering
    
    \includegraphics[width=7.5cm]{pictures/IO_table_2x2_highopen.png}
    \caption{\label{fig:IO2x2_highopen} IO matrix for 2-Country, 2-Sector setup with high openness}
\end{figure}

From the IO matrix (\ref{fig:IO2x2_lowopen}), it is apparent that sector $A\_s1$ has a final demand share of 0.4 and a primary input share of 0.4, and 
sector $A\_s2$ has a final demand share of 0.6 and a primary input share of 0.6. Conversely, sector $B\_s1$ has a final demand share of 0.3 and a 
primary input share of 0.3, and sector $B\_s2$ has a final demand share of 0.5 and a primary input share of 0.5. This IO structure results in 
upstreamness values of
$A\_s1: 2.402,\hspace{3mm} A\_s2: 1.924,\hspace{3mm} B\_s1: 2.616,\hspace{3mm} B\_s2: 2.2$ and downstreamness values of 
$A\_s1: 2.357,\hspace{3mm} A\_s2: 1.921,\hspace{3mm} B\_s1: 2.636,\hspace{3mm} B\_s2: 2.228$. As for the 1-country, 2-sector setup, this structure of low openness (dominant matrix diagonal) has 
created a positive ``correlation'', between upstreamness and downstreamness.
From the IO matrix (\ref{fig:IO2x2_highopen}), it is apparent that sector $A\_s1$ has a final demand share of 0.4 and a primary input share of 0.4, and 
sector $A\_s2$ has a final demand share of 0.6 and a primary input share of 0.6. Conversely, sector $B\_s1$ has a final demand share of 0.3 and a 
primary input share of 0.3, and sector $B\_s2$ has a final demand share of 0.5 and a primary input share of 0.5. This IO structure results in 
upstreamness values of 
$A\_s1: 1.853,\hspace{3mm} A\_s2: 2.264,\hspace{3mm} B\_s1: 2.026,\hspace{3mm} B\_s2: 2.503$ and downstreamness values of 
$A\_s1: 2.481,\hspace{3mm} A\_s2: 2.01,\hspace{3mm} B\_s1: 2.305,\hspace{3mm} B\_s2: 1.85$.  As for the 1-country, 2-sector setup, this structure of 
high openness (dominant matrix diagonal) has created a negative ``correlation'', between upstreamness and downstreamness. based on the evidence provided by 
\textcite{antras2018MeasurementUpstreamnessDownstreamness}, the correlation between upstreamness and downstreamness should increase as openness of the 
international production system increases. Hence, it should be considered critical that these IO tables, set up for a multi-country simulation, do not show 
this pattern. I will discuss the reasons for this pattern and how a future extension of this model might be able to resolve this inconsistency.



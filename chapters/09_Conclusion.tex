\section{Conclusion}
\label{sec:conc}

Throughout the last sections, I have outlined novel empirical facts on the relation between a sector's position on the value chain 
and its price volatility. Concretely, I show that higher upstreamness, representing a stronger exposure to the spillovers of shocks 
to final demand, is associated with higher price volatility. Similarly, I also provide evidence that higher downstreamness,
implying a stronger exposure to the spillovers of shocks to primary input cost, is associated with lower price volatility. I attempt 
to provide some theoretical intuition for the mechanisms behind this fact by constructing a model of N-sector cost minimization
with ``round-about'' Input-Output linkages. I conduct simulations for the upstream propagation of final demand shocks and downstream 
propagation of primary inputs cost shocks for both a 2-Sector and a 2-Country, 2-Sector setup and show that the model simulations 
broadly reproduce the empirical pattern at the first stage of shock transmission. Higher upstreamness is associated with higher 
price volatility, while higher downstreamness is associated with lower price volatility. This pattern is robust over different values for
Returns-To-Scale and simple representations of low and high openness for the Input-Output system. The simulations are run with a short-term 
time in mind. Thus, results should not be considered valid for longer durations since this would give firms the opportunity to adapt 
their production linkages, violating a fundamental assumption of exogenous production structure.

In an attempt to simulate the propagation of shocks along the value chain, separately from propagation over time, I show that in my 
simulated model, the above relations hold in the sectoral cross-section as shock transmission is simulated several stages up 
(for demand shocks) or down (for cost shocks) the production chain. Further theoretical implications show that correlated shocks at 
the country or sector level do not break these general patterns but induce sectoral co-movement as they decrease the differences in 
volatility for a simulated propagation of the shock beyond a first-stage transmission. With the results outlined just now, the model 
shows a strong internal consistency, but it also has some obvious shortcomings, as discussed in section \ref{sec:limit}. However, many of 
these shortcomings have been previously encountered and documented in various contributions to the literature. Hence, a future extension 
of this project with a more complete formal model has the potential to remedy these shortcomings.

Despite the apparent shortcomings, this paper is a valuable contribution to the literature on sectoral price dynamics with 
production linkages. I document two new stylized facts for international monetary economics with production linkages. Furthermore, I 
contribute to understanding upstreamness and downstreamness as measures of shock exposure, a perspective that is underrepresented 
in international macroeconomics. Finally, I provide a simple framework to model how the production linkages that constitute these measures
also drive the propagation of shocks and, hence, give rise to differences in price volatility.

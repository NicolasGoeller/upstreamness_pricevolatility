\section{Data and Stylized Facts}
\label{sec:datafacts}

\subsection{Data Overview}

I use data on sectoral price changes in 31 countries and for an intersection of 17 manufacturing sectors at the ISIC 3.1 classification
for 1995-2011. This dataset was compiled, prepared, and harmonized (e.g. monthly interpolation, currency transformation) 
by \textcite{auer2019InternationalInflationSpillovers} and is freely available for download as a replication package 
\parencite{levchenko2020ReplicationDataInternational}. These price changes are sector-level producer prices (PPI), representing 
prices of ``unfinished'' goods for further use in production (intermediates). This is an important caveat since prices on consumer goods 
(final demand) will be for retail-level quantities, while producer goods are dealt wholesale. Consumer prices also benefit strongly from 
special sales periods, making them potentially more volatile. The data on international Input-Output tables were obtained from the WIOD 2013 
release \parencite{timmer2015IllustratedUserGuide}. Due to the non-availability of price data at appropriate frequency and disaggregation, 
the service sectors will not be included in the analysis.
%that also includes a Socioeconomic Appendix that allows me to compute labor shares explicitly, instead of relying on Value Added shares.

In terms of an overview of the price data, figures (\ref{fig:ppi_chng_density}a) and (\ref{fig:ppi_chng_density}b) show the distribution 
of producer price changes in the cross-section and for each country-sector. It is apparent that the cross-sectional distributions are 
more similar and closer to a normal distribution. Comparetively, the price change distributions for each country-sector are still subject 
to significant heterogeneity. Nonetheless, both plots show very heavy tails and a slightly positive mean. Based on these price changes and 
as defined in section \ref{sec:methods}, I compute the sectoral producer price volatilities for each year as my primary outcome variable. 
The violin plots of the country-sector price volatilities within years, sectors, and countries in appendix \ref{sec:add_viz} show 
significant heterogeneity in the distribution of volatilities within each sector and year. Comparatively, there are only small differences 
between the distributions of volatilities in each year.

\begin{figure}[H]
    \includegraphics[width=\textwidth]{graphs/ppi_chng_density_trunc.png}
    \caption{\label{fig:ppi_chng_density} Density plots over Year-Month and Country-Sector}
\end{figure}

The graphs in figure (\ref{fig:shares_stream}) provide some empirical substance to the intuition of upstreamness and downstreamness being 
measures for indirect shock exposure to final demand or primary input cost, respectively. Here, I show the Pearson correlation coefficient
computed separately for each year of the country-sector values of final demand shares, value added (primary input) share, upstreamness, 
and downstreamness. These graphs generally reproduce the empirical facts shown by \textcite{antras2018MeasurementUpstreamnessDownstreamness},
in the sense of a negative correlation coefficient for both ``upstreamness - final demand share'' and ``downstreamness - value added share''
that weakens over time. Furthermore, the graph shows the same time trend for the correlations upstreamness - downstreamness and final demand
share - value added share. Contrary to the \textcite{antras2018MeasurementUpstreamnessDownstreamness}, these correlations are initially 
negative and become positive only with time. The reason for this should be found in excluding service sectors from my analysis. However,
it is an interesting fact that despite the removal of service sectors, the relations described by 
\textcite{antras2018MeasurementUpstreamnessDownstreamness} remain valid.

\begin{figure}[H]
    \includegraphics[width=\textwidth]{graphs/shares_stream.png}
    \caption{\label{fig:shares_stream} Plots for year-wise correlations for exposure measures}
\end{figure}


%MAYBE INTEGRATION/ OPENNESS OVER TIME - Leave out for now maybe add later

\subsection{Stylized Facts}

Regarding the empirical strength of the relation between sectoral producer price volatility and upstreamness or downstreamness at the 
country-sector level, figure (\ref{fig:macro_corr}) presents an initial overview. This plot shows that the correlation between 
upstreamness and price volatility is positive in the whole period of observation, while the correlation between producer price volatility and 
downstreamness is negative. Both measures are not particularly strong; the correlation between upstreamness and price volatility is around 0.25;
the correlation between downstreamness and price volatility is around -0.1. However, their sign is mostly consistent, and there seems to be a 
trend that aligns with the increasing integration of the global production system. 

\begin{figure}[H]
    \includegraphics[width=\textwidth]{graphs/macro_correlations.png}
    \caption{\label{fig:macro_corr} Correlation of price volatility with upstreamness and downstreamness}
\end{figure}

From this plot alone, it is difficult to judge the validity of the underlying relations since numerous factors common to certain 
countries, sectors, country-sectors, or years might drive the observed positive and negative correlation, respectively. This suspicion 
is consolidated by the violin plots in appendix \ref{sec:add_viz}, showing distributions of producer price volatility within sectors 
and countries. Hence, I conduct regressions with a broad set of fixed effects to test if the described relations between price volatility, 
upstreamness, and downstreamness remain statistically significant. Equation (\ref{eq:regress}) describes the general structure of the 
estimated models, where $\varphi$ is the price volatility, $u$ is upstreamness, $d$ is downstreamness, all indexed by their country-sector 
and year of observation. $\Xi_{i,t}$ denotes the respective fixed effects at country, sector, country-sector, and year levels. Finally, 
$\nu_{i,t}$ is the error term.

\begin{equation} \label{eq:regress}
    \varphi_{i,t} = \beta_u u_{i,t} + \beta_d d_{i,t} + \Xi_{i,t} + \nu_{i,t}
\end{equation}

The results from this equation are displayed in table (\ref{tab:reg_baseline}). Both the coefficients for upstreamness and downstreamness 
are statistically significant across all fixed effect specifications and remain consistent regarding their sign. An interesting observation 
is that this is the case even after the inclusion of country-sector fixed effects.  The robustness of the regression coefficients to 
country-sector fixed effects suggests that the mechanism behind the relation observed in the regression might be a product not only of 
static sector, country, or country-sector characteristics but also a product of dynamic adaptation in the production network. In order to 
further solidify the evidence from these regressions, I have included several more tables of regressions with only upstreamness and 
downstreamness on the RHS and another one with the sum of upstreamness and downstreamness that can be understood as a measure of the length 
of global value chains a country-sector is involved in. Furthermore, I have run the specification shown in equation (\ref{eq:regress}) in a 
year-wise fashion under the addition of just country and sector fixed effects. All of these additional tables can be seen in appendix 
\ref{sec:add_reg}. Generally, the results of these additional regressions remain consistent regarding the sign of the coefficients. However, 
in case of the individual RHS variables, the significance of coefficients does not hold up to the inclusion of fixed effect as well as in 
table (\ref{tab:reg_baseline}). A similar trend is observed for the year-wise regressions, where in the years before 2004, the coefficients tend to lose 
significance under country and sector fixed effects. Given these robustness checks, the results observed in table (\ref{tab:reg_baseline}) could be 
driven by particularly strong values during the years after 2004, partially invalidating the stylized facts presented here. Other possible 
explanations would be a lack of power in the year-wise regression or, again, the relative importance of the dynamic adaptation of production structures. 

\include{tables/reg_ppi_volat_std_base.tex}

%SHOW YEARWISE TABLE (to be produced) - Not for now 


\section{Literature Review}
\label{sec:lit}

This section will outline the relevant research areas that I draw on for this paper and how I contribute to them. First, this includes 
research on how production linkages are established; second, the modeling of the impact of these linkages on fluctuations; third, how 
such models change when applied to prices; and lastly, the specific role of global connections for price dynamics.\\

Input-Output production structures are an object of longstanding interest in economics. The fragmentation of production into individual 
stages distributed domestically and globally gives rise to complicated Input-Output networks. In the literature these are called Global 
Value Chains (GVCs). This representation of production with intermediate inputs along a value chain has allowed researchers to think of 
production processes as upstream or downstream relative to a particular production stage. \textcite{antras2012MeasuringUpstreamnessProduction} 
use this idea of relative positioning in a production network to formalize the notion and measurement of upstreamness and downstreamness of a 
(country-)sector, drawing on the conceptual framework of Leontief Input-Output models. \footnote{see 
\textcite{miller2009InputoutputAnalysisFoundations} for an extensive overview.}  These concepts have subsequently become a standard part of 
research on production structures, understood as an indicator for ``specialization'' in a particular stage of the value chain 
\parencite{wang2017MeasuresParticipationGlobal,alfaro2019InternalizingGlobalValue}. A conceptual refinement of the notion of upstreamness 
was proposed by \textcite{miller2017OutputUpstreamnessInput}, which I will discuss in more detail in section \ref{sec:methods}. 
\textcite{miller2017OutputUpstreamnessInput} also emphasizes the role of upstreamness as an indicator of the ability of a sector to propagate 
a stimulus through the economy, making reference to the discussion around estimating multipliers in the literature on Leontief Input-Output 
models in \textcite[243-250]{miller2009InputoutputAnalysisFoundations}. In parallel to this work on value chain positions, theoretical models 
of international trade have progressed towards explicitly modeling production chains that span multiple countries. An important contribution 
to this literature is \textcite{caliendo2015EstimatesTradeWelfare}, who provide a model of vertical production linkages that can involve 
several countries and sectors. The model is very parsimonious in its required input data and has been widely used as a baseline to evaluate 
different scenarios of trade costs in an environment with GVCs. A particular feature of \textcite{caliendo2015EstimatesTradeWelfare} is 
that value chains are modeled as ``round-about'' implying that the chain's actual length is infinite as country-sectors rely on each 
other's input in a loop structure. Further contributions have made efforts to formalize the formation of global value chains further, 
either as a multi-stage, decentralized production process \parencite{antras2020GeographyGlobalValue} or as a decision about vertical 
integration \parencite{fally2018CoasianModelInternational}. Another feature of \textcite{caliendo2015EstimatesTradeWelfare} is to model input 
sourcing decisions simply as a choice among the cheapest price offered across a specific sector, implying each input can only be sourced from
one country-sector. \textcite{antras2017MarginsGlobalSourcing} provide a model that makes decisions of foreign sourcing interdependent across
the markets for different inputs to account for the unique mechanisms of supplier selection, which are sensitive to substitutability or 
complementarity of inputs, among other things. My contribution to this branch of the literature extends the understanding of upstreamness and 
downstreamness, specifically for applications at the intersection of international trade and macroeconomics. \\
%A particular branch of research strongly
%influenced by International Trade is concerned with the estimation of production links in their extensive and intensive margin.
%For most of this research, the interdependence of trade relations has been absorbed into ``multilateral resistance'' terms in order 
%to make gravity models tractable (see \textcite{head2014ChapterGravityEquations} for a summary). More recently, 
%\textcite{chaney2014NetworkStructureInternational, chaney2018GravityEquationInternational} are the first major contribution to explicitly 
%endogenize this link formation with reference to networks.
%- Bernard \& Moxnes in similar vein
%\textcite{oberfield2018TheoryInputOutput}

Conversely, (international) macroeconomics explores production structures as a determinant of aggregate fluctuations, usually with 
output as the main variable of interest. However, many themes developed for models of output fluctuations have been fruitfully applied 
to price dynamics, as covered in the next paragraph. The foundational contribution of this literature is \textcite{long1983RealBusinessCycles} 
who provided a simple model of business cycle fluctuations with disaggregated sectoral production. One current approach to explain the 
occurrence of aggregate and sectoral volatility is through network transmission of shocks following \textcite{acemoglu2012NetworkOriginsAggregate} 
\parencite{acemoglu2016NetworksMacroeconomyEmpirical,baqaee2018CascadingFailuresProduction,chakrabarti2018DispersionMacroeconomicVolatility}.
This branch of research has remained mostly cross-sectional and focused on ``closed'' economies but is able to model the rich non-linearities 
between shocks and output that result from the network structure. Contrary, following \textcite{basu1995IntermediateGoodsBusiness}, models of 
``International Real Business Cycles'' rely on the correlation of shocks along country or sectoral dimensions as the main explanations of 
country comovement (and hence volatilities) \parencite{burstein2008TradeProductionSharing,atalayHowImportantAre2017}. While this approach 
has a strong intertemporal and international focus, it lacks a more complicated production structure beyond multiple sectors per country. 
In parallel to this theoretical discussion, empirical research by \textcite{digiovanni2009TradeOpennessVolatility, 
digiovanni2010PuttingPartsTogether} has found that trade openness and the strength of vertical production linkages are positively 
affecting aggregate volatility and comovement, respectively. Further work such as \textcite{barrotInputSpecificityPropagation2016}, 
\textcite{digiovanni2018MicroOriginsInternational} has shown that even firm-specific shocks may have a significant impact on aggregate 
fluctuations under certain conditions on input specificity, vertical linkages, and the firm size distribution.\footnote{Much of this 
empirical work is connected to a third branch of theoretical explanations following \textcite{gabaix2011GranularOriginsAggregate} that 
is centered on granular economies. This approach is not discussed at length here since granularity does not play a significant role in 
this project.} Finally, \textcite{joya2019AllSectoralShocks} find that sectors with high centrality in the network of production linkages 
are more likely to propagate shocks that lead to significant aggregate volatility. These contributions demonstrate that network 
transmission and correlated shocks are relevant channels for explaining aggregate fluctuations. Consequently, recent research by 
\textcite{huo2019InternationalComovementGlobal} has tried combining them as complementary channels to explain aggregate fluctuations 
structurally. Based on this framework, \textcite{bonadio2023GlobalizationStructuralChange} built a model of structural change and 
globalization, where they show that reallocation of economic activity from manufacturing to services in many advanced economies has 
produced a pattern in aggregate comovement where the inter-country correlations of GDP have remained mostly stable despite higher 
trade integration. This result is consistent with the fact that shocks to services should be much less correlated across countries than for 
manufacturing. I contribute to this part of the literature as a first step in quantifying the measurement of production chain positions in 
macroeconomic models, a gap specifically brought up by \textcite[27]{smets2019PipelinePressuresSectoral}. From the above literature, 
I am unaware of any use of upstreamness or downstreamness in the macroeconomic models with a short-term interest, such as stabilization 
policy or macroeconomic volatility.\\

In both research areas covered so far, production linkages are discussed in tandem with output. A dedicated literature applies 
modeling concepts on production networks to price dynamics and inflation. This paragraph will discuss the standard facts and models
about price setting and how these are applied to models with IO linkages. In terms of inflation measures, the definition of the 
Consumer Price Index (CPI) is the aggregation of goods prices for consumers, and the Producer Price Index (PPI) is the aggregation 
of goods prices for producers; hence, the distribution of change rates for these inflation measures should be very similar in closed 
economies as shown by \textcite{nakamura2008FiveFactsPrices} and most results discussed in this paragraph hold for CPI and PPI. 
According to \textcite{nakamura2008FiveFactsPrices}, the price change frequency is higher for price increases than decreases and 
covaries positively with inflation. These basic facts can be reproduced with simple menu cost models with Input-Output linkages 
\parencite{nakamura2010MonetaryNonneutralityMultisector}. Here, the presence of intermediate input structures implies a significant 
increase in the impact of aggregate monetary shocks on price setting (monetary non-neutrality). Empirical work has further explored 
the impact of shocks at various levels \parencite{boivin2009StickyPricesMonetary,beck2016ImportanceSectoralRegional} and found 
significant heterogeneity in both the timing and magnitude of the price change response. Concretely, sectoral shocks have an 
immediate but transient effect on price-setting, while responses to aggregate shocks are delayed but more substantial, making 
sectoral responses comparatively more volatile. Subsequently, price-setting models have been able to explain this behavior as a 
combined impact of sectoral labor market segmentation and shock transmission through Input-Output linkages 
\parencite[in circulation since 2011]{carvalho2021SectoralPriceFacts}.\footnote{Other approaches for explanations of this pattern 
are in terms of rational inattention to certain shocks based on \textcite{mackowiakOptimalStickyPrices2009}, or through 
price-setting by multi-product firms, summarised by \textcite{bhattarai2014MultiproductFirmsPricesetting}. These contributions 
will not be discussed here in further detail.} Based on such advances, research on sectoral inflation dynamics and Input-Output 
structures has started to combine New-Keynesian DSGE models with IO linkages. Overall results by 
\textcite{smets2019PipelinePressuresSectoral,afrouzi2023InflationGDPDynamics,minton2023DelayedInflationSupply} suggest that 
sectoral supply shocks propagate downstream and make price rigidities/ persistence accumulate in this process. These papers confirm 
previous findings by \textcite{nakamura2010MonetaryNonneutralityMultisector} that IO linkages increase monetary non-neutrality and 
contribute to price volatility. However, these models remain focused on domestic frameworks (mostly the US) in terms of their 
theoretical mechanism and calibration data. Furthermore, \textcite{smets2019PipelinePressuresSectoral} admits that the 
characterization of upstream or downstream sectors (and propagation direction) in this area of research is primarily ad-hoc in the 
sense of some sector's production being more ``raw'' \parencite[27]{smets2019PipelinePressuresSectoral} than others. In my paper, 
I contribute a simple framework for explaining price volatility through production linkages, similar to these previous contributions.
For this purpose, I reproduce stylized facts between formal measures of production chain position and price volatility and explain them 
through a simple cost optimization framework.\\

%Furthermore, 
%the current discussion about energy price shocks in the context of the Ukraine-Russia war has made the distinction between the CPI 
%and Core inflation (CPI without food and energy) similarly meaningful. Here, \textcite{minton2023DelayedInflationSupply} show that Core Personal Consumption 
%Expenditure (Core PCE) is predictable to a significant degree through a production network model with an oil component. Moreover, energy 
%price shocks also play a significant role in determining pass-through among firms, as shown by \textcite{mejean2023CostPassthroughRise}. 

The previous paragraph has discussed how production linkages affect price dynamics for models in domestic settings. International 
production linkages, as well as the difference between CPI and PPI dynamics, have been mostly ignored in that literature, based on evidence 
from \textcite{nakamura2008FiveFactsPrices}. Indeed, \textcite{weiWedgeCenturyUnderstanding2018} show strong comovement of CPI and PPI 
for most countries over the recording time (positive Pearson correlation coefficient). However, they also show that this comovement 
has weakened with increasing trade integration, and some countries even show a negative correlation in recent years. Previous research
has investigated the global comovement of inflation \parencite{ciccarelli2010GlobalInflation,monacelli2009InternationalDimensionInflation}, 
and found significant global components of inflation. \textcite{auer2019InternationalInflationSpillovers} take a similar approach 
but with the addition of international Input-Output structures for the propagation of cost shocks into the comovement of PPI inflation 
(for a similar contribution on consumer prices see \textcite{auerGlobalisationInflationGrowing2017}). Their results suggest that production 
linkages amplify international comovement significantly, making up about half of the global component of PPI inflation. As such, 
international comovement of inflation is largely not due to common shocks or global trends but the amplification of country, sector, or 
country-sector shocks. Conversely to the previous paragraph, the literature is heavily focussed on the international dimension 
of production linkages and price dynamics. However, there is a lack of papers explicitly explaining this relationship. My contribution 
is a first step in transferring results derived for a domestic setting into a framework that includes international linkages. Since the 
organization of international production systems is highly multidimensional, introducing formal measures of production chain position 
to such models is even more relevant than for domestic frameworks.\\
%The global integration 
%of production and its relation to inflation dynamics has also been a topic of interest after the Covid crisis. Evidence points to a 
%mismatch of supply and demand (termed bottlenecks) in multiple sectors as a driver of the inflation surge during 2021 
%\parencite{labelle2022GlobalSupplyChain,digiovanniGlobalSupplyChain2022}.
%%% Be more explicit
% Out of all the papers presented here, the closest to my contribution are \textcite{afrouzi2023InflationGDPDynamics} and 
% \textcite{auer2019InternationalInflationSpillovers}. However, while the former provides a structural explanation of the impact of 
% production linkages on inflation dynamics, the framework is purely domestic and does not refer to international influences. However, 
% imported inputs are a major feature of modern production systems, and they introduce the possibility for correlated shocks and 
% international Input-Output network propagation. Such evidence is shown by \textcite{auer2019InternationalInflationSpillovers}, where
% production linkages have the potential of strong international propagation of shocks, regardless of their origin in a country, 
% sector, or country-sector. Even so, this evidence derived from a time-series model lacks an explicit theoretical foundation for the 
% impact of production linkages on price dynamics. My contribution aims to fill the gap left between these two papers and provides a 
% theoretical framework that is suited to evaluate the interaction between production linkages and price dynamics at a global scale. 
% Finally, by including a formal definition of upstreamness and downstreamness into the model framework for price dynamics, I provide an 
% opportunity to avoid ad-hoc classifications of upstream and downstream sectors \parencite{smets2019PipelinePressuresSectoral}. Such an 
% addition to the modeling framework has the potential to reveal additional empirical insights about the impact of production linkages on 
% sectoral price dynamics and improve our understanding of the theory behind this relation.
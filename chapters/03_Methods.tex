\section{Methods} \label{sec:methods}

In this section, I define the variables of price volatility, upstreamness, and downstreamness. Following 
\textcite{boivin2009StickyPricesMonetary}, going forward, I define the sectoral volatility of prices $\varphi_i$ as the standard deviation 
of relative price changes (difference of logs) for a sector over an observation period. Intuitively, I write price $p_{i,t}$ for sector $i$ 
at time $t$ and the relative rate of price change as $\Delta p_{i,t}$.%Since these variables are aggregated along some dimension and cannot be attributed to a singular sector at a specific time, their definition and subsequent computation matter to the outcome investigated with them.

\begin{equation}\label{eq:deltap}
    \Delta p_{i,t+1} = ln(p_{i,t+1}) - ln(p_{i,t})
\end{equation}
\begin{equation}\label{eq:def_volat}
    \varphi_i = \sqrt{\frac{1}{T} \sum_{t=1}^T \left( \Delta p_{i,t} - \frac{1}{T} \sum_{t=1}^T \Delta p_{i,t} \right)^2}
\end{equation}

The terms ``upstreamness'' and ``downstreamness'' were first introduced by \textcite{antras2012MeasuringUpstreamnessProduction} as a measure 
to denote a sector's average distance (upstreamness) from final demand in a value chain (downstreamness being the inverse of said distance).
However, \textcite{miller2017OutputUpstreamnessInput} introduce the distinction between output upstreamness (in the sense of 
\textcite{antras2012MeasuringUpstreamnessProduction}) and input downstreamness.\footnote{\textcite{fally2012ProductionStagingMeasurement} 
makes a similar distinction but with less explicit terms.} This refined terminology acknowledges that in a value chain, both the distance from
final demand and the distance from primary inputs (e.g. labor, non-tradeables, services, etc.) matter. Both measures are defined as an 
(infinite) sum over the production stages upstream/ downstream from a sector as written in equation (\ref{eq:updown_series}). In this way, 
upstreamness $u_i$ and downstreamness $d_i$ are characterized by the sectoral output $x_i$, values for final demand $\{f_j\}_{j=1}^N$ and 
primary inputs $\{v_j\}_{j=1}^N$ as well as the values from a matrix $Z$ with sales and purchases of intermediate inputs. Specifically, 
a diagonal matrix with the vector of output values $\hat{X}$ and the matrix of intermediate inputs are combined to derive a matrix $A$ 
with input coefficients $A = Z \hat{X}^{-1}$ ($a_{ij} = z_{ij}/x_j$) and a matrix $B$ with output coefficients 
$B = \hat{X}^{-1} Z$ ($b_{ji} = z_{ji}/x_j$). Respectively, coefficients in $A$ and $B$ represent the share of inputs from sector $i$ 
required for outputs from sector $j$, and the share of outputs from sector $j$ required as inputs by sector $i$. The resulting formulas 
taken from \textcite{miller2017OutputUpstreamnessInput}:

\begin{equation}
    \label{eq:updown_series}
    \begin{split}
        u_i &= 1 \frac{f_i}{x_i} + 2 \frac{\sum_{j} a_{ij} f_j}{x_i} + 3 \frac{\sum_{j,k} a_{ik} a_{kj} f_j}{x_i} + 4 \frac{\sum_{j,k,l} a_{il} a_{lk} a_{kj} f_j}{x_i} + ...\\
        d_i &= 1 \frac{v_i}{x_i} + 2 \frac{\sum_{j} v_j b_{ji}}{x_i} + 3 \frac{\sum_{j,k} v_j b_{jk} b_{ki}}{x_i} + 4 \frac{\sum_{j,k,l} v_j b_{jk} b_{kl} b_{li}}{x_i} + ...
    \end{split}
\end{equation}

The above sector-wise definitions of the measures are equivalent to computationally simpler variants that relate them explicitly to 
Input-Output models. Specifically, \textcite[451-453]{miller2017OutputUpstreamnessInput} show that the equations 
(\ref{eq:updown_series}) can be rewritten in the form of the equations (\ref{eq:updown_linalg}).\footnote{Hence, these definitions of 
upstreamness and downstreamness make these concepts mathematically equivalent to the definitions of total forward and backward linkages
as discussed by \textcite[555-558]{miller2009InputoutputAnalysisFoundations}.} Consequently, upstreamness $U$ can be 
derived as the row sums of the Ghosh Inverse $G$ and downstreamness $D$ as the column sums of the Leontief inverse $L$.
\footnote{These inverses are derived as infinite sum over the power series of their respective coefficient matrices: \begin{itemize}
    \item $L = (I - A)^{-1} = \sum_{n=0}^{\infty} A^n$
    \item $G = (I - B)^{-1} = \sum_{n=0}^{\infty} B^n$
\end{itemize}} Here $\iota$ is a vector of ones with appropriate length.

\begin{equation}
    \label{eq:updown_linalg}
    \begin{split}
        U &=  G \iota = (I - B)^{-1} \iota = (I - \hat{X}^{-1} Z)^{-1} \iota\\
        D &= \iota'L = \iota' (I - A)^{-1} = \iota' (I - Z \hat{X}^{-1})^{-1}
    \end{split}
\end{equation}

As discussed briefly in section \ref{sec:lit}, the literature on production chains in international trade often frames the concept of 
upstreamness as a means to classify the specialization of countries in certain stages of the chain 
\parencite{antras2012MeasuringUpstreamnessProduction}. However, for the interests of macroeconomics, the view proposed by 
\textcite{miller2017OutputUpstreamnessInput} should be more appropriate, where upstreamness is understood as an indicator of the 
ability of a sector to propagate a stimulus through the economy. Specifically, I propose to consider yet another conceptual shift. For 
this paper, understanding upstreamness and downstreamness, respectively, as indicators for exposure to shocks at the end (final demand) 
and start (primary inputs) of the value chain should be conducive. With the final demand share $\frac{f_i}{x_i}$ and the value added 
share $\frac{v_i}{x_i}$ there is already a characterization of direct dependence on these factors. Hence, they express the exposure of 
a sector to direct shocks to its final demand and primary input cost. Considering that the final demand share and value added share, 
respectively, make up the first term of the equations for upstreamness and downstreamness, it is evident that these measures indicate 
the total exposure to shocks propagated through the IO network. In this sense, it is reasonable that direct exposure and total exposure 
are inversely related by design. \textcite{antras2018MeasurementUpstreamnessDownstreamness} confirm this theoretical intuition regarding 
negative correlations between upstreamness and final demand shares, as well as between downstreamness and value added shares. They also 
show that in both cases, from 1995 to 2011, the negative correlation gets weaker over time. While one would expect that sectors close to 
final demand (low upstreamness) should be far away from primary inputs (high downstreamness), there is a positive correlation between the 
two measures (and also between final demand share and value added share). Over the observed period (1995-2011), these correlations are 
found to increase further, making it overall very likely that these observed facts are related to the increasing openness to international 
trade and fragmentation of production. The explanation put forward by \textcite{antras2018MeasurementUpstreamnessDownstreamness} for the 
increasingly positive correlation between upstreamness and downstreamness relates to the structural change towards service sectors 
observed in many economies, since those sectors should have both a high use of primary inputs (e.g. labor), but also attract a large 
share of final consumption. Another candidate explanation is the reduction of trade costs that occurred in this period.

I have referenced Input-Output models multiple times throughout this and the previous section. This term is not simply used for models 
with intermediate input structures but applies to a specific model type, thoroughly discussed by 
\textcite{miller2009InputoutputAnalysisFoundations}. Specifically, the Leontief IO-model is used to evaluate the propagation of 
exogenous changes in final demand, while the Ghosh model does the same for changes in primary inputs. Both models can be formulated in 
terms of physical units or prices (quantity model, price model). By themselves, these models are not commonly used in economics 
publications anymore, although they are a staple in research on regional science and as part of computable general equilibrium models 
(CGEs) for policy evaluation. The basic Input-Output models usually rely on production functions with strict complementarity 
(Leontief-type) and do not have an explicit firm perspective, hence insufficient micro-foundations. Moreover, the propagation in these 
models is commonly interpreted as occurring with physical quantities or prices held constant. Finally, Constant Return-To-Scale as an 
assumption is made for all of these models, implying that all factors of production are flexibly adjustable. This feature, 
together with the derivation of the Leontief and Ghosh inverse $L$, $G$ as infinite sums over the input and output coefficient matrix, 
$A$ and $B$, respectively, give these models a distinct focus on long-term effects, since the infinite iteration over production stages 
and flexibility of production factors only holds for long time horizons. On the other hand, partial equilibrium models are more flexible 
and can be used to explore deviations from several of these assumptions, making them suitable for applications that focus on short-term 
fluctuations.


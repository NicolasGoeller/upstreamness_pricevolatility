\section{Simulation Results}
\label{sec:simres}

In this section, I present and discuss the simulation results for the 1-country, 2-sector, and 2-country, 2-sector models, 
respectively, with downstream transmission of cost shocks and upstream transmission of demand shocks. Each shock scenario will be 
presented through two graph panels of 16 plots each. Each panel of plots has the results from different shock variance parameters on 
the columns and results from different variance values for the initial price changes on the rows. The first graph for each pair 
presents results on the first transmission stage but across several Returns-To-Scale (RTS) indicator values. The second graph for each 
pair shows the propagation of shocks to price volatility across up to five stages but only for one combination of RTS.

\subsection{2-Sector Economy}

Graph (\ref{fig:1x2cost_y1}a) provides the first insights into the one-stage downstream transmission of primary input cost shocks on 
price volatility in equation (\ref{eq:downstream_changes}) for the IO matrix shown in figure (\ref{fig:IO1x2_lowopen}). In this scenario, a more 
downstream sector (sector indicators with downstreamness in legend) will be less volatile (volatilities on y-axis) if the cost shock 
distribution dominates the distribution of initial price changes in variance (upper triangular of plot panel). Generally, the simulated 
volatilities for all sectors and shock scenarios are smaller as the TTS indicator increases (RTS indicators on x-axis). Considering the 
role of the RTS indicator matrix $\Gamma$ in equation (\ref{eq:downstream_changes}), such an effect is mathematically reasonable. 
The theoretical interpretation of this observation can be seen as partial equilibrium adaption with economies of scale in this basic 
model framework. The model intuition suggests that a positive (negative) primary input cost shock or initial price change leads to an 
increase (decrease) in the marginal cost and, hence, buyer price. Since this simulation is conducted under an assumption of perfect 
competition, the price change must be due to a change in the supply produced supply. In this sense, the theoretical role of RTS can 
be understood in terms of the production structure enacting an amplifying (case of decreasing RTS, $\gamma_i < 1$) or diminishing 
(case of increasing RTS, $\gamma_i > 1$) effect on this change. If, as for these simulations, all sectors have the same RTS values, 
it is reasonable to expect decreasing volatilities as RTS values increase. 

Graph (\ref{fig:1x2cost_y1}b) displays the multi-stage downstream propagation in equation (\ref{eq:propagation}) of primary input cost shocks, 
specifically for the case of constant RTS (differences for other RTS are small).\footnote{The volatility values in graph (\ref{fig:1x2cost_y1}a) 
for the RTS indicator equal to one, hence are the same in each panel as the values for stage one volatilities in graph (\ref{fig:1x2cost_y1}b).} 
This graph shows that the pattern of the more downstream sectors being less volatile (volatilities on y-axis) in the cross-section of sectors 
(sector indicators with downstreamness in legend) persists even as the simulation propagates the economy downstream in total (stages 
on the x-axis). Again, as for the one-stage transmission in the previous paragraph, this observation holds only for shock scenarios
where the variance of the cost shock distribution is larger than the variance of the initial price change 
distribution (upper triangular of plot panel). Volatility will also accumulate with downstream propagation stages in the shock scenarios of this 
upper triangular, while it diffuses for the cases in the lower triangular. Furthermore, it is apparent that for the given level 
of RTS, in the lower triangular of graph (\ref{fig:1x2cost_y1}b), the more downstream sector will become less volatile faster and 
even switch the ranking as with propagation into more downstream stages. Plots of the same kind as (\ref{fig:1x2cost_y1}b) but 
for different levels of RTS show that the propagation dynamic will be more pronounced the higher RTS are. The appendix \ref{sec:1x2_sim}
contains the same graphs shown here but for the simulation conducted on the IO matrix in figure (\ref{fig:IO1x2_highopen}). As previously
mentioned, the supposed main difference between the two was the reversal of the ranking of value added shares, implying that this
new IO matrix deviates from the positive correlation between upstreamness and downstreamness shown in the data. Remarkably, both graphs 
do not show any major qualitative differences in the pattern targeted for this project (mind the color change for the more downstream 
sector). 

\begin{figure}[H]
    \includegraphics[width=\textwidth]{graphs/1x2_scenarioplot_cost_shocks_1.0-1.0_year1.png}
    \caption{\label{fig:1x2cost_y1} Cost shock propagation}
\end{figure}

Moving on to final demand shocks, graph (\ref{fig:1x2demand_y1}a) provides the first insights to examine the one-stage upstream transmission
of final demand shocks in equation (\ref{eq:upstream_changes}), again for data simulated on the IO matrix in figure (\ref{fig:IO1x2_lowopen}).
This graph shows that at the first stage of upstream demand shock transmission, more upstream sectors (sector indicators with upstreamness in 
legend) are more volatile (volatilities on y-axis) across all values of RTS (RTS indicators on x-axis), and all combinations of variances for 
final demand shocks and initial price changes. Since there is little difference between the plots in each row but a significant difference 
between those in each column, the variance of demand shocks is of low relative importance compared to the variance of initial price changes.
A higher variance of initial price changes intuitively results in a higher price volatility. For no variance in initial price changes, 
an expected pattern is visible for Constant RTS, where the demand shock has no impact, no matter its variance. Generally and consistent
with the pattern observed for cost shocks before, volatility decreases with RTS across all sectors and in all stage one shock 
scenarios. 

Graph (\ref{fig:1x2demand_y1}b) displays the multi-stage upstream propagation in equation (\ref{eq:propagation}) of final demand shocks, 
specifically for the case of decreasing RTS (differences for other RTS are small).\footnote{The volatility values in graph (\ref{fig:1x2demand_y1}a) 
for the RTS indicator equal to 0.8, hence are the same in each panel as the values for stage one volatilities in graph (\ref{fig:1x2demand_y1}b).}
As already observed for the first stage transmission in graph (\ref{fig:1x2demand_y1}a), more upstream sectors in the cross-section (sector indicators 
with upstreamness in legend) continue to be more volatile (volatilities on y-axis) in all stages (stages on the x-axis) of shock propagation and with 
no qualifications. Furthermore, volatility will only accumulate (or persist) with upstream propagation (see rows 1-2 in graph 
(\ref{fig:1x2demand_y1}b)) if the variance of the initial price change distribution is sufficiently low. This pattern is even more pronounced 
for higher levels of RTS. The appendix \ref{sec:1x2_sim} contains the same graphs as shown here, but for the simulation conducted on the IO matrix in figure 
(\ref{fig:IO1x2_highopen}). Again, the differences between these results are mostly quantitative and do not affect the patterns reported.

\begin{figure}[H]
    \includegraphics[width=\textwidth]{graphs/1x2_scenarioplot_demand_shocks_0.8-0.8_year1.png}
    \caption{\label{fig:1x2demand_y1} Demand shock propagation}
\end{figure}

% Concrete discussion of mechanism
Overall, the patterns observed from the simulated data appear consistent with what has been shown from the empirical data. Sectors with 
higher upstreamness are more volatile in their prices, and sectors with higher downstreamness are less volatile. No major difference 
exists between the simulated results for the IO matrix with ``high openness'' and the one with ``low openness''. This implies that 
the mechanism behind the results for the 1-country, 2-sector setup is all about the sectoral shares of final demand and primary inputs. 
The sector with the higher share of final demand will have lower upstreamness and, hence, lower price volatility. Conversely, the sector
with a higher share of primary inputs will have lower downstreamness and, hence, higher price volatility. This emphasizes the importance of
understanding how the direct shock exposure reflected in these shares gives rise to both the production chain positions and the
price volatility. As an indicator of direct shock exposures, the shares determine the exposures to indirect shocks and, hence, the 
aggregate relations observable in the data. This model grossly oversimplifies the situation and most likely overemphasizes 
the direct shares' role and the impact of purely country-sector shocks in determining price volatility. Since such a simplification 
is necessary to explain the underlying mechanisms but the dangers of losing external validity, the next section is a step toward 
qualifying this simplification. 


\subsection{2-Country, 2-Sector Economy}

The first stage downstream transmission of cost shocks seen in graph (\ref{fig:2x2cost_i_y1}a) provides insights on how shocks to primary input 
costs drive differences in price volatility with two countries and two sectors. The simulations shown here are conducted for the IO 
matrix in figure (\ref{fig:IO2x2_lowopen}). As before, the x-axis shows RTS values, the y-axis volatilities, and the legend country-sector
indicators with downstreamness values. The basic result is consistent with the one from the 1-country,2-sector 
setup in the previous section, and shows that more downstream sectors are less volatile in prices. The differences in simulated price 
volatility are more pronounced for higher variances of cost shock. This overall pattern is reversed only if cost shock variance is very 
low compared to the variance of initial price changes, which is in contrast to the 1x2 IO matrix simulation, where reversal was observed 
for equal variance values. This is most likely a result of the higher share of value added in this 2x2 IO matrix compared to the 1x2 
IO matrix. Otherwise, it appears to be the case still that higher RTS values cause lower simulated volatilities and that 
higher variances of initial price change cause higher simulated volatilities at this first stage of shock transmission. 

Graph (\ref{fig:2x2cost_i_y1}b) displays the multi-stage downstream propagation in equation (\ref{eq:propagation}) of primary input cost shocks, 
specifically for the case of constant RTS (differences for other RTS are small).\footnote{The volatility values in graph (\ref{fig:2x2cost_i_y1}a) 
for the RTS indicator equal to 1.0, hence are the same in each panel as the values for stage one volatilities in graph (\ref{fig:2x2cost_i_y1}b).}
The overall pattern 
between volatility and downstreamness in the country-sector cross-section persists as the shock propagates downstream along the value 
chain. Moreover, volatility accumulates with propagation down the chain for higher variance values (graph (\ref{fig:2x2cost_i_y1}b), 
columns 3 and 4) but is diffused otherwise (columns 1 and 2). These results are also found again for simulations of the second IO matrix 
with country-sector shocks, visualized in the appendix \ref{sec:2x2_sim}. Appendix \ref{sec:2x2_sim} also contains the graph for country-sector
shocks to primary input cost for the simulations on IO matrix (\ref{fig:IO2x2_highopen}), the graphs for other shocks were omitted since they 
would not have added much to this paper. Furthermore, I find that cost shocks 
originating purely at the country or sector level do not change these simulated patterns significantly. However, they do induce a 
convergence of volatilities for the multi-stage shock propagation, and they weaken the dominance of cost shocks over initial price 
changes in terms of impact on volatility (as observed in graph (\ref{fig:2x2cost_i_y1}a)). Finally, the simulations under shock scenarios 
that mix country with country-sector shocks and sector with country-sector shocks show some more complicated patterns.

\begin{figure}[H]
    \includegraphics[width=\textwidth]{graphs/2x2_scenarioplot_i_cost_shocks_RS0.8_year1.png}
    \caption{\label{fig:2x2cost_i_y1} Country-sector cost shock propagation}
\end{figure}

Moving on to final demand shocks, graph (\ref{fig:2x2demand_i_y1}a) provides insights into how shocks to final demand drive differences in 
price volatility in the first stage of upstream transmission. The simulations shown here are conducted for the IO 
matrix in figure (\ref{fig:IO2x2_lowopen}). As before, the x-axis shows RTS values, the y-axis volatilities, and the legend country-sector
indicators with upstreamness values. More upstream sectors are also more volatile across all values 
for RTS and variance of shocks and initial price change. In general, volatility decreases as RTS increase, but 
it also appears to be the case that the initial price changes are more impactful for volatilities than the cost shock. This is 
evident from the increase in the level of volatility with the rows in graph (\ref{fig:2x2demand_i_y1}a), as well as the minor differences 
only between the graph's columns. Finally, the patterns for Constant RTS that are visible for zero variance of initial 
price changes do not reappear for the other plot rows. All of this is very consistent with the observed patterns from results for the 
1-country, 2-sector setup.

Graph (\ref{fig:2x2demand_i_y1}b) displays the multi-stage upstream propagation in equation (\ref{eq:propagation}) of final demand shocks, 
specifically for the case of decreasing RTS (differences for other RTS are small).\footnote{The volatility values in graph (\ref{fig:2x2demand_i_y1}a) 
for the RTS indicator equal to 0.8, hence are the same in each panel as the values for stage one volatilities in graph (\ref{fig:2x2demand_i_y1}b).}
The general relation of upstreamness and price volatility in the 
country-sector cross-section discussed just now also persists for the propagation up the value chain as shown in graph (\ref{fig:2x2demand_i_y1}b). 
Again, the dominance of the initial price changes is apparent here since volatility diffuses as the shocks propagate up the chain. This 
diffusion of volatility occurs for plot rows with non-zero price change variances (graph (\ref{fig:2x2demand_i_y1}b), rows 2-4), and 
accumulation only happens if initial price changes have zero variance (row 1). These results are also found again with one exception 
for simulations of the second IO matrix (\ref{fig:IO2x2_highopen}) with country-sector shocks, visualized in the appendix \ref{sec:2x2_sim}. 
The graphs for other shocks to the setup of IO matrix (\ref{fig:IO2x2_highopen}) were omitted since they would not have added much to this paper.
This exception is that at the first stage, the two most upstream sectors do not have a significant difference in volatility despite 
having a difference in upstreamness. However, as the shock propagates up the value chain, this is corrected to be consistent with the 
expected pattern. Comparing the 2-Country, 2-Sector setup, and the 1-Country, 2-Sector setup, results on the impact from country-sector shocks to 
final demand are highly similar. Even the introduction of correlated shocks at the country and sector level or with combined shock 
origins (country and country-sector, sector and country-sector) do not bring significant changes in the simulated patterns. Most likely, 
this is related to the generally weak impact of final demand shocks relative to the initial price change in this framework.


\begin{figure}[H]
    \includegraphics[width=\textwidth]{graphs/2x2_scenarioplot_i_demand_shocks_RS0.8_year1.png}
    \caption{\label{fig:2x2demand_i_y1} Country-sector demand shock propagation}
\end{figure}

% Concrete discussion of mechanism is contrast to 2-sector example

Overall, the patterns described for the simulations on the 2-Country, 2-Sector setup are still consistent with the stylized facts 
found in the data. The introduction of correlated shocks at the country or sector level as stand-alone scenarios does not 
cause a major break in the pattern. Correlated shocks at the country or sector level appear to drive a ``comovement'' of prices that 
emerges over a decrease in the difference between country-sector volatilities for the muti-stage propagation plots. This is a sensible pattern since 
shocks that affect a set of country-sectors equally should cause a convergence by themselves already. But moreover, this convergence should be 
accelerated through the IO linkages between country-sector during propagation up or down the value chain. Furthermore, there is no significant difference 
to observe in qualitative terms between the IO matrix setup for high and low openness. The continued stability of the pattern, as empirically observed 
in the data, is a sign of strong internal consistency of 
the model behind these patterns. The shares of final demand and primary input against total output appear to be the main drivers 
since they determine the direct shock exposure and the total (indirect) exposure (upstreamness, downstreamness) through the 
network as well. As such, the model equations (\ref{eq:propagation}) provide a formal framework of how shocks to final demand (primary input cost)
affect country-sectors in a production system directly and indirectly through the network. This is visible already for the first-stage transmission 
but becomes even more apparent as shocks are simulated to propagate up (down) the production chain over multiple stages. However, at this point, 
the lack of an observable difference in results between the IO matrices with high and low ``openness''
raises the question if this framework might be over relying on the mechanism around final demand shares and primary input shares. 
This might be an interesting challenge for a future extension of the model, as discussed in section \ref{sec:limit}.

%From the overall discussion of the results for the 2-Country, 2-Sector setup and the comparison to the 2-Sector setup, the patterns are 
%remarkably consistent. This indicates that the model outlined in this paper has an internally consistent mechanism that is robust to 
%changes in its inputs . 

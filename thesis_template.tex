\documentclass[12pt]{article}
\usepackage[utf8]{inputenc}
\usepackage[T1]{fontenc}
\usepackage{graphicx}
\usepackage{subfig}
\usepackage[english]{babel}
\usepackage[a4paper, width=150mm, top=25mm, bottom=25mm, left=25mm, right=25mm]{geometry}
\usepackage{authblk}
\usepackage{fancyhdr}
\pagestyle{fancy}
\fancyhf{}
\fancyhead[LE,RO]{}
\fancyfoot[LE,RO]{\hfill\thepage\hfill}
\usepackage[onehalfspacing]{setspace}


\usepackage{csquotes}
\usepackage[style=authoryear-icomp, maxcitenames=2, backend=biber]{biblatex}
\usepackage[colorlinks=true, citecolor=., linkcolor=., urlcolor=blue]{hyperref}
\addbibresource{masterthesis.bib}

% Some generic settings.
\newcommand{\cmd}[1]{\texttt{\textbackslash #1}}
\setlength{\parindent}{0pt}
\newenvironment{bibsample}
  {\trivlist\samepage
   \setlength{\itemsep}{0pt}}
  {\endtrivlist}

%%To make a Subtitle
\usepackage{titling}
\newcommand{\subtitle}[1]{%
\posttitle{%
    \par\end{center}
\begin{center}\large#1\end{center}
\vskip0.5em}%
}

\newcommand{\cm}[3][\mathord]{%
  #1{\text{\textcolor{#2}{$#3$}}}%
}%% originally it was \text{\mathversion{black}...

\newcommand{\bb}[1]{\mathbb{#1}}

\usepackage{appendix}

\DeclareUnicodeCharacter{1E62}{S}

%Tables
\usepackage{multirow}
\usepackage{adjustbox}
\usepackage{array}
\usepackage{booktabs}
\usepackage{siunitx}
%%Equations
\usepackage{amsmath}
\usepackage{amssymb}
%%Figures
\usepackage{float}

% Colors
\usepackage{xcolor}



\begin{document}

\begin{titlepage}
\title{Producer Price Volatility Up and Down the Value Chain\\[1ex]
  \large A Bullwhip-Effect for Prices?} 
\begin{center}
\large PSE Master Thesis
\vfill
   \includegraphics[width=7.5cm]{pictures/PSE logo.png}
\vfill   
\end{center}
\author{Nicolas Göller \thanks{Paris School of Economics \href{mailto:nicolas.goeller@etu.univ-paris1.fr}
{nicolas.goeller@etu.univ-paris1.fr}. I want to thank my supervisor Ariell Reshef for patient and insightful
 research guidance. The paper has also benefitted from discussions with Angelo Secchi, Mathieu Parenti, Frederic Denker, Mudigonda Sarath Chandra, Alice Mazzacurati, and Nicholas Tokay as well as with the participants of the APE Macroeconomics Research Seminar.
 I also thank the ``Studienstiftung des Deutschen Volkes'' for its financial support.}}
\date{\today}
\maketitle
\begin{abstract}
\noindent This study investigates the relationship between a country-sector's position in the production chain and its price volatility using a theoretical model and numerical simulations. The model's results, validated against empirical data, show that sectors positioned farther from final demand (upstream) experience higher price volatility, while those farther from primary inputs (downstream) exhibit lower volatility. This pattern for a cross-section of country-sectors holds even as shocks are simulated to propagate further up and down the production chain. Correlated shocks at country or sector levels do not disrupt this pattern but lead to sectoral comovement. These findings contribute to our understanding of the relation between international production linkages and price dynamics.
\\
\vspace{0in}\\
\noindent\textbf{Keywords:} sectoral price volatility, input-output linkages, upstreamness\\
\vspace{0in}\\
\noindent\textbf{JEL Codes:} E31, E32, F15, F66\\
\vspace{1.7in}\\
\\

\bigskip
\end{abstract}
\thispagestyle{empty}
\end{titlepage}

%%% Make tables of content, figures and tables
\tableofcontents
\setcounter{page}{0}
\thispagestyle{empty}
\newpage
\listoffigures
\setcounter{page}{0}
\thispagestyle{empty}
\listoftables
\newpage

\section{Introduction}\
\label{sec:intro}

% What is the motivation
Within the last three years, the world economy has experienced its first episode of significant price fluctuations since the Global 
Financial Crisis in 2008. These price dynamics were induced first by the Covid crisis and then by the energy spill-overs from the 
Russia-Ukraine war. Research has shown that supply chain linkages have significantly impacted the increased volatility of prices. Evidence, 
in particular, suggests that the impact of an economic crisis is channeled from a few strongly affected sectors into the broader economy
through supply chain linkages. \textcite{labelle2022GlobalSupplyChain} document this for technology equipment and automobiles during 
the Covid crisis and \textcite{minton2023DelayedInflationSupply} do so for Petroleum Refining in case of the Russia-Ukraine war. In a 
third line of inquiry \textcite{amiti2021HighImportPrices} show that the prices of inputs imported to the US reacted strongly, in 
particular for unfinished industrial inputs during the Covid crisis (see figure (\ref{fig:covid_imports})). 

\begin{figure}[H]
    \centering
    \includegraphics[width=0.9\textwidth]{graphs/amiti_intro_plot_prices.png}
    \caption{\label{fig:covid_imports} Prices for Imported Goods in the US as taken from \textcite{amiti2021HighImportPrices}}
\end{figure}

Naturally, this event is unique in the size of its shocks and their global scale. With global supply chains spanning countries and 
sectors, price dynamics could be specific to a country, a sector in several countries, or a particular country-sector combination. Anecdotally,
however, for a coarse sectoral disaggregation, and for the US only, evidence as in figure (\ref{fig:covid_imports}) suggests
that sectors conventionally considered more upstream in the value chain are more volatile in their prices than downstream ones, especially since 
2021, but potentially even before. Such an observation appears to be in parallel to phenomena such as the ``Bullwhip effect'' commonly discussed in 
management science \parencite{sterman1989ModelingManagerialBehavior}. Consequently, this combination of anecdotal evidence and thematic commonality 
with a related discipline raises a general question about the relationship between the position of a country-sector in the 
global supply network and the volatility of its prices: Does the position of a country-sector along the value chain matter for its price 
volatility?

% why is this important?
The comovement and volatility of prices are covered by a significant body of literature that includes production systems with intermediate 
inputs from theoretical and empirical perspectives. Consequently, this paper is a theoretical exploration, specifically about the implication 
of a country-sector's position in the production chain toward price volatility. Within studies of Input-Output systems, such positions 
include the country-sector's distance to final demand and local input supply, respectively, and the diversification among its buyers and 
suppliers. My analysis's results should constitute a mapping from production chain position to price volatility and how various types of 
shocks affect this relationship.

In order to obtain a theoretically founded mapping of production chain positions to price volatility, I derive a model for N-sector cost 
minimization with Input-Output structures. This model simulates price outcomes for the upstream propagation of final demand shocks   
and the downstream propagation of primary input cost shocks. The proposed model and its numerically simulated results for both a 
1-Country, 2-Sector setup, and a 2-Country, 2-Sector setup reproduce the descriptive evidence derived through linear regression analysis
on empirical data for sectoral prices and IO linkages. The simulated model and empirical analysis show that as a country-sector's distance 
to final demand (upstreamness) increases, so does its price volatility. Similarly, as a country-sector's distance to primary inputs 
(downstreamness) increases, its price volatility should be observed to decrease. The first of these results is well in line with figure 
(\ref{fig:covid_imports}), where industrial supply prices change much stronger than consumer goods prices. The second result is harder to 
verify with this graph due to the difficulty in determining distances to primary inputs for the depicted sectors. In my model, both 
these relations hold even as shock transmission is simulated over several stages up (for final demand shocks) or down (for primary 
input cost shocks) the production chain. Further theoretical implications show that correlated shocks at the country or sector level 
do not break this general pattern but induce sectoral comovement as they decrease the differences in volatility for a simulated 
propagation of the shocks beyond a first-stage transmission.

% what am I doing?
The rest of the paper is structured as follows. Section \ref{sec:lit} provides an overview of the relevant literature, while section 
\ref{sec:methods}, \ref{sec:datafacts}, and \ref{sec:model} will discuss the methods, the stylized facts and the model and simulation 
framework for this paper, respectively. Section \ref{sec:simres} presents the results derived from the simulations under various 
Input-Output structures and their implications. Finally, section \ref{sec:limit} discusses the limitations of the approach utilized in 
this paper but also possible extensions for future research, and section \ref{sec:conc} concludes.


\section{Literature Review}
\label{sec:lit}

This section will outline the relevant research areas that I draw on for this paper and how I contribute to them. First, this includes 
research on how production linkages are established; second, the modeling of the impact of these linkages on fluctuations; third, how 
such models change when applied to prices; and lastly, the specific role of global connections for price dynamics.\\

Input-Output production structures are an object of longstanding interest in economics. The fragmentation of production into individual 
stages distributed domestically and globally gives rise to complicated Input-Output networks. In the literature these are called Global 
Value Chains (GVCs). This representation of production with intermediate inputs along a value chain has allowed researchers to think of 
production processes as upstream or downstream relative to a particular production stage. \textcite{antras2012MeasuringUpstreamnessProduction} 
use this idea of relative positioning in a production network to formalize the notion and measurement of upstreamness and downstreamness of a 
(country-)sector, drawing on the conceptual framework of Leontief Input-Output models. \footnote{see 
\textcite{miller2009InputoutputAnalysisFoundations} for an extensive overview.}  These concepts have subsequently become a standard part of 
research on production structures, understood as an indicator for ``specialization'' in a particular stage of the value chain 
\parencite{wang2017MeasuresParticipationGlobal,alfaro2019InternalizingGlobalValue}. A conceptual refinement of the notion of upstreamness 
was proposed by \textcite{miller2017OutputUpstreamnessInput}, which I will discuss in more detail in section \ref{sec:methods}. 
\textcite{miller2017OutputUpstreamnessInput} also emphasizes the role of upstreamness as an indicator of the ability of a sector to propagate 
a stimulus through the economy, making reference to the discussion around estimating multipliers in the literature on Leontief Input-Output 
models in \textcite[243-250]{miller2009InputoutputAnalysisFoundations}. In parallel to this work on value chain positions, theoretical models 
of international trade have progressed towards explicitly modeling production chains that span multiple countries. An important contribution 
to this literature is \textcite{caliendo2015EstimatesTradeWelfare}, who provide a model of vertical production linkages that can involve 
several countries and sectors. The model is very parsimonious in its required input data and has been widely used as a baseline to evaluate 
different scenarios of trade costs in an environment with GVCs. A particular feature of \textcite{caliendo2015EstimatesTradeWelfare} is 
that value chains are modeled as ``round-about'' implying that the chain's actual length is infinite as country-sectors rely on each 
other's input in a loop structure. Further contributions have made efforts to formalize the formation of global value chains further, 
either as a multi-stage, decentralized production process \parencite{antras2020GeographyGlobalValue} or as a decision about vertical 
integration \parencite{fally2018CoasianModelInternational}. Another feature of \textcite{caliendo2015EstimatesTradeWelfare} is to model input 
sourcing decisions simply as a choice among the cheapest price offered across a specific sector, implying each input can only be sourced from
one country-sector. \textcite{antras2017MarginsGlobalSourcing} provide a model that makes decisions of foreign sourcing interdependent across
the markets for different inputs to account for the unique mechanisms of supplier selection, which are sensitive to substitutability or 
complementarity of inputs, among other things. My contribution to this branch of the literature extends the understanding of upstreamness and 
downstreamness, specifically for applications at the intersection of international trade and macroeconomics. \\
%A particular branch of research strongly
%influenced by International Trade is concerned with the estimation of production links in their extensive and intensive margin.
%For most of this research, the interdependence of trade relations has been absorbed into ``multilateral resistance'' terms in order 
%to make gravity models tractable (see \textcite{head2014ChapterGravityEquations} for a summary). More recently, 
%\textcite{chaney2014NetworkStructureInternational, chaney2018GravityEquationInternational} are the first major contribution to explicitly 
%endogenize this link formation with reference to networks.
%- Bernard \& Moxnes in similar vein
%\textcite{oberfield2018TheoryInputOutput}

Conversely, (international) macroeconomics explores production structures as a determinant of aggregate fluctuations, usually with 
output as the main variable of interest. However, many themes developed for models of output fluctuations have been fruitfully applied 
to price dynamics, as covered in the next paragraph. The foundational contribution of this literature is \textcite{long1983RealBusinessCycles} 
who provided a simple model of business cycle fluctuations with disaggregated sectoral production. One current approach to explain the 
occurrence of aggregate and sectoral volatility is through network transmission of shocks following \textcite{acemoglu2012NetworkOriginsAggregate} 
\parencite{acemoglu2016NetworksMacroeconomyEmpirical,baqaee2018CascadingFailuresProduction,chakrabarti2018DispersionMacroeconomicVolatility}.
This branch of research has remained mostly cross-sectional and focused on ``closed'' economies but is able to model the rich non-linearities 
between shocks and output that result from the network structure. Contrary, following \textcite{basu1995IntermediateGoodsBusiness}, models of 
``International Real Business Cycles'' rely on the correlation of shocks along country or sectoral dimensions as the main explanations of 
country comovement (and hence volatilities) \parencite{burstein2008TradeProductionSharing,atalayHowImportantAre2017}. While this approach 
has a strong intertemporal and international focus, it lacks a more complicated production structure beyond multiple sectors per country. 
In parallel to this theoretical discussion, empirical research by \textcite{digiovanni2009TradeOpennessVolatility, 
digiovanni2010PuttingPartsTogether} has found that trade openness and the strength of vertical production linkages are positively 
affecting aggregate volatility and comovement, respectively. Further work such as \textcite{barrotInputSpecificityPropagation2016}, 
\textcite{digiovanni2018MicroOriginsInternational} has shown that even firm-specific shocks may have a significant impact on aggregate 
fluctuations under certain conditions on input specificity, vertical linkages, and the firm size distribution.\footnote{Much of this 
empirical work is connected to a third branch of theoretical explanations following \textcite{gabaix2011GranularOriginsAggregate} that 
is centered on granular economies. This approach is not discussed at length here since granularity does not play a significant role in 
this project.} Finally, \textcite{joya2019AllSectoralShocks} find that sectors with high centrality in the network of production linkages 
are more likely to propagate shocks that lead to significant aggregate volatility. These contributions demonstrate that network 
transmission and correlated shocks are relevant channels for explaining aggregate fluctuations. Consequently, recent research by 
\textcite{huo2019InternationalComovementGlobal} has tried combining them as complementary channels to explain aggregate fluctuations 
structurally. Based on this framework, \textcite{bonadio2023GlobalizationStructuralChange} built a model of structural change and 
globalization, where they show that reallocation of economic activity from manufacturing to services in many advanced economies has 
produced a pattern in aggregate comovement where the inter-country correlations of GDP have remained mostly stable despite higher 
trade integration. This result is consistent with the fact that shocks to services should be much less correlated across countries than for 
manufacturing. I contribute to this part of the literature as a first step in quantifying the measurement of production chain positions in 
macroeconomic models, a gap specifically brought up by \textcite[27]{smets2019PipelinePressuresSectoral}. From the above literature, 
I am unaware of any use of upstreamness or downstreamness in the macroeconomic models with a short-term interest, such as stabilization 
policy or macroeconomic volatility.\\

In both research areas covered so far, production linkages are discussed in tandem with output. A dedicated literature applies 
modeling concepts on production networks to price dynamics and inflation. This paragraph will discuss the standard facts and models
about price setting and how these are applied to models with IO linkages. In terms of inflation measures, the definition of the 
Consumer Price Index (CPI) is the aggregation of goods prices for consumers, and the Producer Price Index (PPI) is the aggregation 
of goods prices for producers; hence, the distribution of change rates for these inflation measures should be very similar in closed 
economies as shown by \textcite{nakamura2008FiveFactsPrices} and most results discussed in this paragraph hold for CPI and PPI. 
According to \textcite{nakamura2008FiveFactsPrices}, the price change frequency is higher for price increases than decreases and 
covaries positively with inflation. These basic facts can be reproduced with simple menu cost models with Input-Output linkages 
\parencite{nakamura2010MonetaryNonneutralityMultisector}. Here, the presence of intermediate input structures implies a significant 
increase in the impact of aggregate monetary shocks on price setting (monetary non-neutrality). Empirical work has further explored 
the impact of shocks at various levels \parencite{boivin2009StickyPricesMonetary,beck2016ImportanceSectoralRegional} and found 
significant heterogeneity in both the timing and magnitude of the price change response. Concretely, sectoral shocks have an 
immediate but transient effect on price-setting, while responses to aggregate shocks are delayed but more substantial, making 
sectoral responses comparatively more volatile. Subsequently, price-setting models have been able to explain this behavior as a 
combined impact of sectoral labor market segmentation and shock transmission through Input-Output linkages 
\parencite[in circulation since 2011]{carvalho2021SectoralPriceFacts}.\footnote{Other approaches for explanations of this pattern 
are in terms of rational inattention to certain shocks based on \textcite{mackowiakOptimalStickyPrices2009}, or through 
price-setting by multi-product firms, summarised by \textcite{bhattarai2014MultiproductFirmsPricesetting}. These contributions 
will not be discussed here in further detail.} Based on such advances, research on sectoral inflation dynamics and Input-Output 
structures has started to combine New-Keynesian DSGE models with IO linkages. Overall results by 
\textcite{smets2019PipelinePressuresSectoral,afrouzi2023InflationGDPDynamics,minton2023DelayedInflationSupply} suggest that 
sectoral supply shocks propagate downstream and make price rigidities/ persistence accumulate in this process. These papers confirm 
previous findings by \textcite{nakamura2010MonetaryNonneutralityMultisector} that IO linkages increase monetary non-neutrality and 
contribute to price volatility. However, these models remain focused on domestic frameworks (mostly the US) in terms of their 
theoretical mechanism and calibration data. Furthermore, \textcite{smets2019PipelinePressuresSectoral} admits that the 
characterization of upstream or downstream sectors (and propagation direction) in this area of research is primarily ad-hoc in the 
sense of some sector's production being more ``raw'' \parencite[27]{smets2019PipelinePressuresSectoral} than others. In my paper, 
I contribute a simple framework for explaining price volatility through production linkages, similar to these previous contributions.
For this purpose, I reproduce stylized facts between formal measures of production chain position and price volatility and explain them 
through a simple cost optimization framework.\\

%Furthermore, 
%the current discussion about energy price shocks in the context of the Ukraine-Russia war has made the distinction between the CPI 
%and Core inflation (CPI without food and energy) similarly meaningful. Here, \textcite{minton2023DelayedInflationSupply} show that Core Personal Consumption 
%Expenditure (Core PCE) is predictable to a significant degree through a production network model with an oil component. Moreover, energy 
%price shocks also play a significant role in determining pass-through among firms, as shown by \textcite{mejean2023CostPassthroughRise}. 

The previous paragraph has discussed how production linkages affect price dynamics for models in domestic settings. International 
production linkages, as well as the difference between CPI and PPI dynamics, have been mostly ignored in that literature, based on evidence 
from \textcite{nakamura2008FiveFactsPrices}. Indeed, \textcite{weiWedgeCenturyUnderstanding2018} show strong comovement of CPI and PPI 
for most countries over the recording time (positive Pearson correlation coefficient). However, they also show that this comovement 
has weakened with increasing trade integration, and some countries even show a negative correlation in recent years. Previous research
has investigated the global comovement of inflation \parencite{ciccarelli2010GlobalInflation,monacelli2009InternationalDimensionInflation}, 
and found significant global components of inflation. \textcite{auer2019InternationalInflationSpillovers} take a similar approach 
but with the addition of international Input-Output structures for the propagation of cost shocks into the comovement of PPI inflation 
(for a similar contribution on consumer prices see \textcite{auerGlobalisationInflationGrowing2017}). Their results suggest that production 
linkages amplify international comovement significantly, making up about half of the global component of PPI inflation. As such, 
international comovement of inflation is largely not due to common shocks or global trends but the amplification of country, sector, or 
country-sector shocks. Conversely to the previous paragraph, the literature is heavily focussed on the international dimension 
of production linkages and price dynamics. However, there is a lack of papers explicitly explaining this relationship. My contribution 
is a first step in transferring results derived for a domestic setting into a framework that includes international linkages. Since the 
organization of international production systems is highly multidimensional, introducing formal measures of production chain position 
to such models is even more relevant than for domestic frameworks.\\
%The global integration 
%of production and its relation to inflation dynamics has also been a topic of interest after the Covid crisis. Evidence points to a 
%mismatch of supply and demand (termed bottlenecks) in multiple sectors as a driver of the inflation surge during 2021 
%\parencite{labelle2022GlobalSupplyChain,digiovanniGlobalSupplyChain2022}.
%%% Be more explicit
% Out of all the papers presented here, the closest to my contribution are \textcite{afrouzi2023InflationGDPDynamics} and 
% \textcite{auer2019InternationalInflationSpillovers}. However, while the former provides a structural explanation of the impact of 
% production linkages on inflation dynamics, the framework is purely domestic and does not refer to international influences. However, 
% imported inputs are a major feature of modern production systems, and they introduce the possibility for correlated shocks and 
% international Input-Output network propagation. Such evidence is shown by \textcite{auer2019InternationalInflationSpillovers}, where
% production linkages have the potential of strong international propagation of shocks, regardless of their origin in a country, 
% sector, or country-sector. Even so, this evidence derived from a time-series model lacks an explicit theoretical foundation for the 
% impact of production linkages on price dynamics. My contribution aims to fill the gap left between these two papers and provides a 
% theoretical framework that is suited to evaluate the interaction between production linkages and price dynamics at a global scale. 
% Finally, by including a formal definition of upstreamness and downstreamness into the model framework for price dynamics, I provide an 
% opportunity to avoid ad-hoc classifications of upstream and downstream sectors \parencite{smets2019PipelinePressuresSectoral}. Such an 
% addition to the modeling framework has the potential to reveal additional empirical insights about the impact of production linkages on 
% sectoral price dynamics and improve our understanding of the theory behind this relation. %Prod networks, volatility, price dynamics, supply chains
\section{Methods} \label{sec:methods}

In this section, I define the variables of price volatility, upstreamness, and downstreamness. Following 
\textcite{boivin2009StickyPricesMonetary}, going forward, I define the sectoral volatility of prices $\varphi_i$ as the standard deviation 
of relative price changes (difference of logs) for a sector over an observation period. Intuitively, I write price $p_{i,t}$ for sector $i$ 
at time $t$ and the relative rate of price change as $\Delta p_{i,t}$.%Since these variables are aggregated along some dimension and cannot be attributed to a singular sector at a specific time, their definition and subsequent computation matter to the outcome investigated with them.

\begin{equation}\label{eq:deltap}
    \Delta p_{i,t+1} = ln(p_{i,t+1}) - ln(p_{i,t})
\end{equation}
\begin{equation}\label{eq:def_volat}
    \varphi_i = \sqrt{\frac{1}{T} \sum_{t=1}^T \left( \Delta p_{i,t} - \frac{1}{T} \sum_{t=1}^T \Delta p_{i,t} \right)^2}
\end{equation}

The terms ``upstreamness'' and ``downstreamness'' were first introduced by \textcite{antras2012MeasuringUpstreamnessProduction} as a measure 
to denote a sector's average distance (upstreamness) from final demand in a value chain (downstreamness being the inverse of said distance).
However, \textcite{miller2017OutputUpstreamnessInput} introduce the distinction between output upstreamness (in the sense of 
\textcite{antras2012MeasuringUpstreamnessProduction}) and input downstreamness.\footnote{\textcite{fally2012ProductionStagingMeasurement} 
makes a similar distinction but with less explicit terms.} This refined terminology acknowledges that in a value chain, both the distance from
final demand and the distance from primary inputs (e.g. labor, non-tradeables, services, etc.) matter. Both measures are defined as an 
(infinite) sum over the production stages upstream/ downstream from a sector as written in equation (\ref{eq:updown_series}). In this way, 
upstreamness $u_i$ and downstreamness $d_i$ are characterized by the sectoral output $x_i$, values for final demand $\{f_j\}_{j=1}^N$ and 
primary inputs $\{v_j\}_{j=1}^N$ as well as the values from a matrix $Z$ with sales and purchases of intermediate inputs. Specifically, 
a diagonal matrix with the vector of output values $\hat{X}$ and the matrix of intermediate inputs are combined to derive a matrix $A$ 
with input coefficients $A = Z \hat{X}^{-1}$ ($a_{ij} = z_{ij}/x_j$) and a matrix $B$ with output coefficients 
$B = \hat{X}^{-1} Z$ ($b_{ji} = z_{ji}/x_j$). Respectively, coefficients in $A$ and $B$ represent the share of inputs from sector $i$ 
required for outputs from sector $j$, and the share of outputs from sector $j$ required as inputs by sector $i$. The resulting formulas 
taken from \textcite{miller2017OutputUpstreamnessInput}:

\begin{equation}
    \label{eq:updown_series}
    \begin{split}
        u_i &= 1 \frac{f_i}{x_i} + 2 \frac{\sum_{j} a_{ij} f_j}{x_i} + 3 \frac{\sum_{j,k} a_{ik} a_{kj} f_j}{x_i} + 4 \frac{\sum_{j,k,l} a_{il} a_{lk} a_{kj} f_j}{x_i} + ...\\
        d_i &= 1 \frac{v_i}{x_i} + 2 \frac{\sum_{j} v_j b_{ji}}{x_i} + 3 \frac{\sum_{j,k} v_j b_{jk} b_{ki}}{x_i} + 4 \frac{\sum_{j,k,l} v_j b_{jk} b_{kl} b_{li}}{x_i} + ...
    \end{split}
\end{equation}

The above sector-wise definitions of the measures are equivalent to computationally simpler variants that relate them explicitly to 
Input-Output models. Specifically, \textcite[451-453]{miller2017OutputUpstreamnessInput} show that the equations 
(\ref{eq:updown_series}) can be rewritten in the form of the equations (\ref{eq:updown_linalg}).\footnote{Hence, these definitions of 
upstreamness and downstreamness make these concepts mathematically equivalent to the definitions of total forward and backward linkages
as discussed by \textcite[555-558]{miller2009InputoutputAnalysisFoundations}.} Consequently, upstreamness $U$ can be 
derived as the row sums of the Ghosh Inverse $G$ and downstreamness $D$ as the column sums of the Leontief inverse $L$.
\footnote{These inverses are derived as infinite sum over the power series of their respective coefficient matrices: \begin{itemize}
    \item $L = (I - A)^{-1} = \sum_{n=0}^{\infty} A^n$
    \item $G = (I - B)^{-1} = \sum_{n=0}^{\infty} B^n$
\end{itemize}} Here $\iota$ is a vector of ones with appropriate length.

\begin{equation}
    \label{eq:updown_linalg}
    \begin{split}
        U &=  G \iota = (I - B)^{-1} \iota = (I - \hat{X}^{-1} Z)^{-1} \iota\\
        D &= \iota'L = \iota' (I - A)^{-1} = \iota' (I - Z \hat{X}^{-1})^{-1}
    \end{split}
\end{equation}

As discussed briefly in section \ref{sec:lit}, the literature on production chains in international trade often frames the concept of 
upstreamness as a means to classify the specialization of countries in certain stages of the chain 
\parencite{antras2012MeasuringUpstreamnessProduction}. However, for the interests of macroeconomics, the view proposed by 
\textcite{miller2017OutputUpstreamnessInput} should be more appropriate, where upstreamness is understood as an indicator of the 
ability of a sector to propagate a stimulus through the economy. Specifically, I propose to consider yet another conceptual shift. For 
this paper, understanding upstreamness and downstreamness, respectively, as indicators for exposure to shocks at the end (final demand) 
and start (primary inputs) of the value chain should be conducive. With the final demand share $\frac{f_i}{x_i}$ and the value added 
share $\frac{v_i}{x_i}$ there is already a characterization of direct dependence on these factors. Hence, they express the exposure of 
a sector to direct shocks to its final demand and primary input cost. Considering that the final demand share and value added share, 
respectively, make up the first term of the equations for upstreamness and downstreamness, it is evident that these measures indicate 
the total exposure to shocks propagated through the IO network. In this sense, it is reasonable that direct exposure and total exposure 
are inversely related by design. \textcite{antras2018MeasurementUpstreamnessDownstreamness} confirm this theoretical intuition regarding 
negative correlations between upstreamness and final demand shares, as well as between downstreamness and value added shares. They also 
show that in both cases, from 1995 to 2011, the negative correlation gets weaker over time. While one would expect that sectors close to 
final demand (low upstreamness) should be far away from primary inputs (high downstreamness), there is a positive correlation between the 
two measures (and also between final demand share and value added share). Over the observed period (1995-2011), these correlations are 
found to increase further, making it overall very likely that these observed facts are related to the increasing openness to international 
trade and fragmentation of production. The explanation put forward by \textcite{antras2018MeasurementUpstreamnessDownstreamness} for the 
increasingly positive correlation between upstreamness and downstreamness relates to the structural change towards service sectors 
observed in many economies, since those sectors should have both a high use of primary inputs (e.g. labor), but also attract a large 
share of final consumption. Another candidate explanation is the reduction of trade costs that occurred in this period.

I have referenced Input-Output models multiple times throughout this and the previous section. This term is not simply used for models 
with intermediate input structures but applies to a specific model type, thoroughly discussed by 
\textcite{miller2009InputoutputAnalysisFoundations}. Specifically, the Leontief IO-model is used to evaluate the propagation of 
exogenous changes in final demand, while the Ghosh model does the same for changes in primary inputs. Both models can be formulated in 
terms of physical units or prices (quantity model, price model). By themselves, these models are not commonly used in economics 
publications anymore, although they are a staple in research on regional science and as part of computable general equilibrium models 
(CGEs) for policy evaluation. The basic Input-Output models usually rely on production functions with strict complementarity 
(Leontief-type) and do not have an explicit firm perspective, hence insufficient micro-foundations. Moreover, the propagation in these 
models is commonly interpreted as occurring with physical quantities or prices held constant. Finally, Constant Return-To-Scale as an 
assumption is made for all of these models, implying that all factors of production are flexibly adjustable. This feature, 
together with the derivation of the Leontief and Ghosh inverse $L$, $G$ as infinite sums over the input and output coefficient matrix, 
$A$ and $B$, respectively, give these models a distinct focus on long-term effects, since the infinite iteration over production stages 
and flexibility of production factors only holds for long time horizons. On the other hand, partial equilibrium models are more flexible 
and can be used to explore deviations from several of these assumptions, making them suitable for applications that focus on short-term 
fluctuations.

 %Volatility, Down/Upstreamness & Models with IO and GE dimensions
\section{Data and Stylized Facts}
\label{sec:datafacts}

\subsection{Data Overview}

I use data on sectoral price changes in 31 countries and for an intersection of 17 manufacturing sectors at the ISIC 3.1 classification
for 1995-2011. This dataset was compiled, prepared, and harmonized (e.g. monthly interpolation, currency transformation) 
by \textcite{auer2019InternationalInflationSpillovers} and is freely available for download as a replication package 
\parencite{levchenko2020ReplicationDataInternational}. These price changes are sector-level producer prices (PPI), representing 
prices of ``unfinished'' goods for further use in production (intermediates). This is an important caveat since prices on consumer goods 
(final demand) will be for retail-level quantities, while producer goods are dealt wholesale. Consumer prices also benefit strongly from 
special sales periods, making them potentially more volatile. The data on international Input-Output tables were obtained from the WIOD 2013 
release \parencite{timmer2015IllustratedUserGuide}. Due to the non-availability of price data at appropriate frequency and disaggregation, 
the service sectors will not be included in the analysis.
%that also includes a Socioeconomic Appendix that allows me to compute labor shares explicitly, instead of relying on Value Added shares.

In terms of an overview of the price data, figures (\ref{fig:ppi_chng_density}a) and (\ref{fig:ppi_chng_density}b) show the distribution 
of producer price changes in the cross-section and for each country-sector. It is apparent that the cross-sectional distributions are 
more similar and closer to a normal distribution. Comparetively, the price change distributions for each country-sector are still subject 
to significant heterogeneity. Nonetheless, both plots show very heavy tails and a slightly positive mean. Based on these price changes and 
as defined in section \ref{sec:methods}, I compute the sectoral producer price volatilities for each year as my primary outcome variable. 
The violin plots of the country-sector price volatilities within years, sectors, and countries in appendix \ref{sec:add_viz} show 
significant heterogeneity in the distribution of volatilities within each sector and year. Comparatively, there are only small differences 
between the distributions of volatilities in each year.

\begin{figure}[H]
    \includegraphics[width=\textwidth]{graphs/ppi_chng_density_trunc.png}
    \caption{\label{fig:ppi_chng_density} Density plots over Year-Month and Country-Sector}
\end{figure}

The graphs in figure (\ref{fig:shares_stream}) provide some empirical substance to the intuition of upstreamness and downstreamness being 
measures for indirect shock exposure to final demand or primary input cost, respectively. Here, I show the Pearson correlation coefficient
computed separately for each year of the country-sector values of final demand shares, value added (primary input) share, upstreamness, 
and downstreamness. These graphs generally reproduce the empirical facts shown by \textcite{antras2018MeasurementUpstreamnessDownstreamness},
in the sense of a negative correlation coefficient for both ``upstreamness - final demand share'' and ``downstreamness - value added share''
that weakens over time. Furthermore, the graph shows the same time trend for the correlations upstreamness - downstreamness and final demand
share - value added share. Contrary to the \textcite{antras2018MeasurementUpstreamnessDownstreamness}, these correlations are initially 
negative and become positive only with time. The reason for this should be found in excluding service sectors from my analysis. However,
it is an interesting fact that despite the removal of service sectors, the relations described by 
\textcite{antras2018MeasurementUpstreamnessDownstreamness} remain valid.

\begin{figure}[H]
    \includegraphics[width=\textwidth]{graphs/shares_stream.png}
    \caption{\label{fig:shares_stream} Plots for year-wise correlations for exposure measures}
\end{figure}


%MAYBE INTEGRATION/ OPENNESS OVER TIME - Leave out for now maybe add later

\subsection{Stylized Facts}

Regarding the empirical strength of the relation between sectoral producer price volatility and upstreamness or downstreamness at the 
country-sector level, figure (\ref{fig:macro_corr}) presents an initial overview. This plot shows that the correlation between 
upstreamness and price volatility is positive in the whole period of observation, while the correlation between producer price volatility and 
downstreamness is negative. Both measures are not particularly strong; the correlation between upstreamness and price volatility is around 0.25;
the correlation between downstreamness and price volatility is around -0.1. However, their sign is mostly consistent, and there seems to be a 
trend that aligns with the increasing integration of the global production system. 

\begin{figure}[H]
    \includegraphics[width=\textwidth]{graphs/macro_correlations.png}
    \caption{\label{fig:macro_corr} Correlation of price volatility with upstreamness and downstreamness}
\end{figure}

From this plot alone, it is difficult to judge the validity of the underlying relations since numerous factors common to certain 
countries, sectors, country-sectors, or years might drive the observed positive and negative correlation, respectively. This suspicion 
is consolidated by the violin plots in appendix \ref{sec:add_viz}, showing distributions of producer price volatility within sectors 
and countries. Hence, I conduct regressions with a broad set of fixed effects to test if the described relations between price volatility, 
upstreamness, and downstreamness remain statistically significant. Equation (\ref{eq:regress}) describes the general structure of the 
estimated models, where $\varphi$ is the price volatility, $u$ is upstreamness, $d$ is downstreamness, all indexed by their country-sector 
and year of observation. $\Xi_{i,t}$ denotes the respective fixed effects at country, sector, country-sector, and year levels. Finally, 
$\nu_{i,t}$ is the error term.

\begin{equation} \label{eq:regress}
    \varphi_{i,t} = \beta_u u_{i,t} + \beta_d d_{i,t} + \Xi_{i,t} + \nu_{i,t}
\end{equation}

The results from this equation are displayed in table (\ref{tab:reg_baseline}). Both the coefficients for upstreamness and downstreamness 
are statistically significant across all fixed effect specifications and remain consistent regarding their sign. An interesting observation 
is that this is the case even after the inclusion of country-sector fixed effects.  The robustness of the regression coefficients to 
country-sector fixed effects suggests that the mechanism behind the relation observed in the regression might be a product not only of 
static sector, country, or country-sector characteristics but also a product of dynamic adaptation in the production network. In order to 
further solidify the evidence from these regressions, I have included several more tables of regressions with only upstreamness and 
downstreamness on the RHS and another one with the sum of upstreamness and downstreamness that can be understood as a measure of the length 
of global value chains a country-sector is involved in. Furthermore, I have run the specification shown in equation (\ref{eq:regress}) in a 
year-wise fashion under the addition of just country and sector fixed effects. All of these additional tables can be seen in appendix 
\ref{sec:add_reg}. Generally, the results of these additional regressions remain consistent regarding the sign of the coefficients. However, 
in case of the individual RHS variables, the significance of coefficients does not hold up to the inclusion of fixed effect as well as in 
table (\ref{tab:reg_baseline}). A similar trend is observed for the year-wise regressions, where in the years before 2004, the coefficients tend to lose 
significance under country and sector fixed effects. Given these robustness checks, the results observed in table (\ref{tab:reg_baseline}) could be 
driven by particularly strong values during the years after 2004, partially invalidating the stylized facts presented here. Other possible 
explanations would be a lack of power in the year-wise regression or, again, the relative importance of the dynamic adaptation of production structures. 

\include{tables/reg_ppi_volat_std_base.tex}

%SHOW YEARWISE TABLE (to be produced) - Not for now 

 %Regressions with FE for basic facts, maybe year-wise
\section{Multisector Model with Intermediate Inputs}
\label{sec:model}

After the basic concepts for this paper have been defined in the previous sections, I will now outline a basic model framework of cost 
minimization. I utilize this model, heavily inspired by \textcite{caliendo2015EstimatesTradeWelfare}, to provide an intuition to explain 
the effect of shocks on final demand and cost of primary inputs under a partial equilibrium in the goods market. Simulations of upstream 
and downstream propagation can be conducted with this framework for various Input-Output structures and shock scenarios. This should 
generate an understanding of the relationship between the position of a sector in the value chain and its price volatility. The previous 
explanation of upstreamness and downstreamness, this model outline, and the specification of toy Input-Output tables will all follow the 
same notational conventions regarding intermediate inputs, input coefficients, total output, and final demand. 

\subsection{Model Outline for Cost Minimization}

With an N-sector economy, the corresponding optimization problem of a representative firm in industry $i$ contains $N+1$ factor inputs 
(N intermediates $\{z_{ji}\}_{j=1}^N $, one local primary input $l_i$, e.g. labor, services) and their prices 
$w, \hspace{3pt}\{p_{j}\}_{j=1}^N$. As such, this implies the existence of N+1 markets for factor inputs, all of which are assumed to 
be perfectly competitive for this paper. For simplicity, a Cobb-Douglas production function is used. This functional form is a strong 
assumption since it imposes constant shares for all factors of production, hence an exogenous production structure. However, it also 
implies an elasticity of substitution between the production factors equal to one, which implies strong substitutability of inputs. 
This is a significant deviation from the Leontief production functions, usually employed for Input-Output models. Hence, the model 
constructed here might be expected to yield propagation results that are more conservative than a similar Leontief-type model.
Finally, a Cobb-Douglas production also implies that optimal production will have to employ inputs at equal orders of magnitude for 
all inputs with non-zero input shares. This assumption is increasingly unrealistic as the number of factors rises. I refer to the 
factors shares for this production function as input shares for the primary $\alpha_{0i}$  and intermediate $\{\alpha_{ji}\}_{j=1}^N$ 
inputs. The matrix $A$ of these intermediate input shares is equivalent to the matrix $A$ of input coefficients in section 
\ref{sec:methods} and hence constitutes the Input-Output structure for this model. I also include a labor productivity variable 
$\tau_i$ that applies to primary inputs. All variables are denoted with a singular sector index, e.g. $i$, but this can be 
easily extended for country-sector indices. I write the corresponding minimization problem for cost $C_i$ as follows:

\begin{equation} \label{eq:min_prob}
\begin{split}
    &\min_{\substack{l_i, \{m_{ji}\}_{j=1}^N}} \mspace{5mu} C_i = wl_i + \sum_{j=1}^N p_j z_{ji} \\
   s.t. \mspace{10mu} &x_i = (\tau_i l_i)^{\alpha_{0i}} \prod_{j=1}^N z_{ji}^{\alpha_{ji}}
\end{split}
\end{equation}

For notational simplicity, in the following I will abbreviate $Z_i = \prod_{j=1}^N z_{ji}^{\alpha_{ji}}$. Furthermore, it must be noted
that in most of the literature, the sum of input shares $\sum_{j=0}^N \alpha_{ji} = \gamma_i$ is assumed to be equal to 1, implying 
constant returns-to-scale. However, this project will explicitly not make this assumption due to the short-term interest of the 
question asked and hence explore the impact of different returns-to-scale (as compared to classical IO models mentioned in section 
\ref{sec:methods}). Consequently, one can write the N+1 FOCs of the above optimisation problem, where $\mu_i$ is the Lagrange multiplier:

\begin{equation} \label{eq:FOC}
\begin{split}
    & w - \mu_i\alpha_{0i}\tau_i^{\alpha_{0i}}l_i^{\alpha_{0i} -1} Z_i =0\\
    & p_j - \mu_i(\tau_i l_i)^{\alpha_{0i}} \alpha_{ji} z_{ji}^{-1} Z_i = 0 \hspace{10pt} \forall j
\end{split}
\end{equation}

This, in turn, allows me to derive the MRTS of factor inputs and their conditional factor demands from substituting into the
production function.

\begin{equation} \label{eq:CFD}
\begin{split}
    & l_i =  x_i^{\frac{1}{\gamma_i}} \left(\frac{\alpha_{0i}}{w \tau_i} \right)^{\frac{\gamma_i -\alpha_{0i}}{\gamma_i}} \prod_{j=1}^N \left(\frac{p_j}{\alpha_{ji}}\right)^{\frac{\alpha_{ji}}{\gamma_i}} \\
    & z_{ji} =  x_i^{\frac{1}{\gamma_i}} \left(\frac{w}{\alpha_{0i}\tau_i}\right)^{\frac{\alpha_{0i}}{\gamma_i}} \frac{\alpha_{ji}}{p_j} \prod_{k=1}^N \left( \frac{p_k}{\alpha_{ki}} \right)^{\frac{\alpha_{ki}}{\gamma_i}}
\end{split}
\end{equation}

By inserting these factor demands into the cost function, I obtain its final formulation and can derive by output $x_i$ for the marginal
cost function with marginal cost $mc_i$ on the LHS. This function is written with the variables demand $x_i$, primary input unit cost 
$\frac{w}{\tau_i}$, intermediate input prices $\{p_j\}_{j=1}^N$, and has the input shares $\{\alpha_{ji}\}_{j=0}^N$ and returns-to-scale 
indicator $\gamma_i$ as parameters. Going forward, I also denote the primary input unit cost as $\omega_i = \frac{w}{\tau_i}$.
\footnote{
The term $\Theta_i$ denotes a firm-specific (sector-specific since I assume representative firms) constant. Specifically, 
$\Theta_i = \prod_{j=0}^N \left(\frac{1}{\alpha_{ji}}\right)^{\frac{\alpha_{ji}}{\gamma_i}}$ This notation can be found in a similar 
manner with \textcite[footnote 19]{caliendo2015EstimatesTradeWelfare}.}

\begin{equation} \label{eq:cost_func}
\begin{split}
    C_i &= w l_i + \sum_{j=1}^N p_j z_{ji} \\
    C_i &= \gamma_i x_i^{\frac{1}{\gamma_i}} \left(\omega_i\right)^{\frac{\alpha_{0i}}{\gamma_i}} \prod_{j=1}^N p_j^{\frac{\alpha_{ji}}{\gamma_i}} \Theta_i\\
    mc_i =\frac{\partial C_i}{\partial x_i} &= x_i^{\frac{1 - \gamma_i}{\gamma_i}} \left(\omega_i \right)^{\frac{\alpha_{0i}}{\gamma_i}} \prod_{j=1}^N p_j^{\frac{\alpha_{ji}}{\gamma_i}} \Theta_i
\end{split}
\end{equation}

In order to observe variations in this price-setting mechanism, I introduce the potential for shocks to final demand and labor cost. 
The demand shock requires output to be decomposed into a sum final demand and intermediates sales $x_i = \sum_{j=1}^N z_{ij} + f_i = S_i + f_i$. 
This decomposition is a relevant feature of the model since the theoretical framework has a strong connection to the share of final demand 
in total output $\lambda_i$ that should also be reflected in the model's structure. For simplicity, it is assumed that the shocks are drawn 
with a mean zero from a normal distribution and apply to an exogenous constant, which is the final demand component and the effective wage 
in this framework. Thus, this way of constructing shocks will produce random fluctuations of the shocked variable around its partial 
equilibrium state. I will consider shocks to final demand and to the local primary input's unit cost. The shock to the primary input's 
unit cost is defined in a log-normal manner for equation (\ref{eq:cost_shock}), as is standard in the literature. As for the shock to final
demand, the additive decomposition of demand necessitates that I adopt a slightly different approach, seen in 
equation (\ref{eq:demand_shock}), since I need to take the logs of this demand component at a later stage in equation 
(\ref{eq:demand_change}).\footnote{Numerically, there is a difference between this way of writing the demand shock and doing so in the more 
standard, log-normal fashion. However, the difference is quite small, and the analytical layout remains reasonably easy to interpret. Thus, 
this approach was deemed acceptable, given the benefits for the model shock structure at later stages.} Technically, this demand shock could 
make the final demand negative ($\eta_i < -1$), leading to theoretically implausible values. I exclude this possibility by setting a logical 
condition of minimal shock values in my code. Hence, I write as follows:

\begin{equation} \label{eq:demand_shock}
    x_i = S_i + f_i (1+\eta_i),  \hspace{10pt} \eta_i \sim \mathcal{N}(0,\sigma_{\eta})
    %x_i = S_i + f_i e^{\eta_i},  \hspace{10pt} \eta_i \sim \mathcal{N}(0,\sigma_{\eta})
\end{equation}
\begin{equation} \label{eq:cost_shock}
    \omega_i =  \frac{w}{\tau_i} e^{\varepsilon_i}, \hspace{10pt} \varepsilon_i \sim \mathcal{N}(0,\sigma_{\varepsilon})
\end{equation}

With this, the final marginal cost function with shocks writes:
\begin{equation}\label{eq:mc_cost}
    p_i = mc_i = \left( S_i + f_i (1+\eta_i) \right)^{\frac{1- \gamma_i}{\gamma_i}} \left(\frac{w}{\tau_i} e^{\varepsilon_i} \right)^{\frac{\alpha_{0i}}{\gamma_i}} \prod_{j=1}^N p_j^{\frac{\alpha_{ji}}{\gamma_i}} \Theta_i
    %mc_i = \left( f_i e^{\eta_i} + S_i \right)^{\frac{1- \gamma_i}{\gamma_i}} \left(\frac{w}{\tau_i} e^{\varepsilon_i} \right)^{\frac{\alpha_{0i}}{\gamma_i}} \prod_{j=1}^N p_j^{\frac{\alpha_{ji}}{\gamma_i}} \prod_{j=0}^N \left(\frac{1}{\alpha_{ji}}\right)^{\frac{\alpha_{ji}}{\gamma_i}}
\end{equation}

Under the assumptions that (1) price setting is simultaneous across sectors, (2) inputs are sourced and used at prices set in the 
previous period, (3) the input-output structure is exogenous, (4) pass-through is complete (e.g. lack of price rigidities, etc.),  
(5) competition is perfect, and (6) each sector consists of one representative firm, it is possible to derive simulated, sectoral price 
series from equation (\ref{eq:mc_cost}). Among these assumptions, (1) is trivial since firms do not wait for other firms from other 
sectors to set their price. Assumption (2) has to be understood in the context of the partial equilibrium model framework, where 
deviations of sectoral prices can exist at different stages of the production chain. Finally, assumptions (3), (4), (5), and (6) are 
acceptable at this point because they make working with this framework tractable without impeding the clarity of the mechanism to be 
shown here. Especially, assumption (5) on perfect competition is to be seen critically here since I intentionally do not assume 
constant Returns-To-Scale. However, under non-constant Returns-To-Scale, it is unlikely for competition to be strong enough to push 
profits to zero and hence validate this assumption. Given that the non-shock terms of equation (\ref{eq:mc_cost}) are derived as a result
of optimization under partial equilibrium, it may possible to see these shock responses as stochastic fluctuations around an optimal state that 
balance out in the time series average. Nonetheless, this is clearly imperfect, ignoring the potential for profits and especially 
profit-seeking firm behavior. I further discuss how one could relax this and the other assumptions in section \ref{sec:limit} and what 
this might imply for the results. 


In applying the equation (\ref{eq:mc_cost}) on itself, each vector of output prices is the vector of input prices in the next 
iteration. However, this marginal cost equation is extremely sensitive to the magnitude of price levels, making the overall simulation 
process very dependent on the ``equilibrium prices''. Such prices will be derived by any system of production linkages (and associated 
primary input costs, demands, productivities, etc.) after several iterations without any exogenous shocks (draws of 
$\eta_i, \varepsilon_i$) at reasonable accuracy. The sensitivity of price setting to the level of input prices reflects the well-known 
empirical fact about differences between price distributions across different countries and sectors (primarily their first moment). 
Since the distributions of price changes are much more comparable across countries and sectors, a model of the theoretical problem 
above but written in price changes might be more appropriate to represent the intended phenomenon. Regarding the formulation of price 
changes in equation (\ref{eq:deltap}), the first step towards this new model is to write the marginal cost in logs and with a time 
subscript (while taking out the shock decomposition for notational simplicity).

%Since volatility is defined as the variance of the relative price changes observed over a particular period, the model has to produce a 
%simulated output in form of a time series prices to compute price changes. If I can make the assumption that price setting is 
%simultaneous across sectors, but inputs are used at prices produced in the previous time period, then the input prices and the prices
%implied by the marginal cost will potentially be different. Then it is possible, to obtain a time series of simulated prices, by 
%iterating equation (\ref{eq:mc_cost}) multiple times, making each output of prices the input prices in the next iteration. If the structure
%of the production network remains unchanged over the simulation (assumed exogenous), any differences in prices should be due to shocks
%to the sector, its direct upstream suppliers and the higher-order upstream suppliers in the previous period(s).
%It is important to note 
%here that this equation is extremely sensitive to the magnitude of price level, making the overall simulation process very dependent on 
%the ``equilibrium''. Any system of production linkages (and associated wages, demands, productivities, etc.) will yield a vector of 
%``equilibrium'' that this function will mechanically arrive at after several iterations without any exogenous shocks ($\eta_i, 
%\varepsilon_i$). This dependence of the level-based simulation process corresponds to the large differences between empirically observed price 
%distributions across different countries and sectors (primarily their first moment). However, the distributions of price changes are 
%much more comparable across countries and sectors. A model of the theoretical problem above but written in price changes might hence be 
%more appropriate to represent the intended phenomenon. In reference to the formulation of price changes in (\ref{eq:deltap}), the first 
%step towards this new model is to write the marginal cost in logs and with a time subscript (while taking out the shock decomposition 
%for notational simplicity).

%It is important to note here that this equation is extremely sensitive to the magnitude of price level, making the overall simulation 
%process very dependent on the ``equilibrium''. Any system of production linkages (and associated wages, demands, productivities, etc.)
%will yield a vector of ``equilibrium'' that this function will mechanically arrive at after several iterations without any 
%exogenous shocks. If the starting price is sufficiently far away from equilibrium, the new price will be influenced moreso by the move 
%towards ``equilibrium'' than the response to shocks. For this reason, I allow for several iterations of this cost function until all 
%price changes of the iteration are below a ``convergence threshold''. Since I am  strictly interested in the accumulation/ dispersion of 
%shocks through a system of production linkages, it is necessary to start introducing shocks with a vector of initial prices at or near 
%this ``equilibrium''. 


\begin{equation} \label{eq:log_mc}
    ln(p_{i,t}) = ln(mc_{i,t}) = \frac{1-\gamma_i}{\gamma_i} ln(x_{i,t}) + \frac{\alpha_{0i}}{\gamma_i} ln(\omega_{i,t}) + \sum_{j=1}^N \frac{\alpha_{ji}}{\gamma_i} ln(p_{i,t-1}) + ln(\Theta_i)
\end{equation}

Recalling equation (\ref{eq:mc_cost}), I can write the relative price change $\Delta p_{i,t+1}$ with this log marginal cost:

\begin{equation}
    \begin{split}
        \Delta p_{i,t+1} &= ln(p_{i,t+1}) - ln(p_{i,t}) \\
        &= \frac{1-\gamma_i}{\gamma_i} [ln(x_{i,t+1}) - ln(x_{i,t})] + \frac{\alpha_{0i}}{\gamma_i} [ln(\omega_{i,t+1}) - ln(\omega_{i,t})] + \\ 
        & \sum_{j=1}^N \frac{\alpha_{ji}}{\gamma_i} [ln(p_{i,t}) - ln(p_{i,t-1})] \\
        &= \frac{1-\gamma_i}{\gamma_i} \Delta x_{i,t+1} + \frac{\alpha_{0i}}{\gamma_i} \Delta \omega_{i,t+1} + \sum_{j=1}^N \frac{\alpha_{ji}}{\gamma_i} \Delta p_{i,t} 
    \end{split}
\end{equation}

In order to complete the reformulation as a price change equation, I now re-consider the shocks to demand (\ref{eq:demand_shock}) and 
primary input cost (\ref{eq:cost_shock}), and rewrite them as relative changes. Since $\omega_i$ has a simple multiplicative structure, the 
changes result trivially in the absolute change of the shock draws, denoted by $d \varepsilon_{i,t+1}$. However, in the case of reformulating 
the demand shock, intermediate sales as a constant necessitate more extensive steps. After normalizing the demand on the sum of intermediate 
sales and constant final demand, the remaining term of $ln(1+x)$ can be written as an infinite geometric sum through the Taylor expansion. 
The last step holds as a first-order approximation of the actual value at the first order. Given that the shocks are drawn from the same stationary 
distribution $\eta_i \sim \mathcal{N}(0,\sigma_{\eta})$ at relatively small variance values, the likelihood of cases to occur where this
approximation fails grossly is considered small. Hence, it was deemed appropriate. For simplicity of notation, I write the share of final 
demand in output as $\frac{f_i}{(S_i+f_i)} = \lambda_i$. The results can now be written as the product of final demand share and the absolute 
change in final demand shock $\lambda_i d \eta_{i,t+1}$. This would not have been possible under a log-normal shock formulation, thus justifying 
the deviation from the standard in the literature. 

\begin{equation} \label{eq:cost_change}
    \begin{split}
        \Delta \omega_{i,t+1} &= ln(\omega_{i,t+1}) - ln(\omega_{i,t}) \\
        &= ln(\frac{w}{\tau_i} e^{\varepsilon_{i,t+1}}) - ln(\frac{w}{\tau_i} e^{\varepsilon_{i,t}}) \\
        &= ln(\frac{w}{\tau_i}) + ln(e^{\varepsilon_{i,t+1}}) - ln(\frac{w}{\tau_i}) - ln(e^{\varepsilon_{i,t}}) \\
        &= \varepsilon_{i,t+1} - \varepsilon_{i,t} \\
        &= d \varepsilon_{i,t+1}
    \end{split}
\end{equation}

\begin{equation} \label{eq:demand_change}
    \begin{split}
        \Delta x_{i,t+1} &= ln(x_{i,t+1}) - ln(x_{i,t}) \\
        &= ln(S_i + f_i(1+\eta_{i,t+1})) - ln(S_i + f_i(1+\eta_{i,t}))\\
        &= ln\left((1 + \frac{f_i}{S_i + f_i}\eta_{i,t+1}) (S_i+f_i) \right) - ln\left((1 + \frac{f_i}{(S_i+f_i)} \eta_{i,t}) (S_i+f_i) \right) \\
        &= ln(1 + \frac{f_i}{(S_i+f_i)} \eta_{i,t+1}) + ln((S_i+f_i)) - ln(1 + \frac{f_i}{(S_i+f_i)} \eta_{i,t}) - ln((S_i+f_i)) \\
        &= ln(1 + \frac{f_i}{(S_i+f_i)} \eta_{i,t+1}) - ln(1 + \frac{f_i}{(S_i+f_i)} \eta_{i,t}) \\
        &= \sum_{n=1}^{\infty} \frac{(-1)^{n+1}}{n} \left(\frac{f_i}{(S_i+f_i)} \eta_{i,t+1}\right)^n - \sum_{n=1}^{\infty} \frac{(-1)^{n+1}}{n} \left(\frac{f_i}{(S_i+f_i)} \eta_{i,t}\right)^n \\
        &= \sum_{n=1}^{\infty} \frac{(-1)^{n+1}}{n} \left(\frac{f_i}{(S_i+f_i)}\right)^n (\eta_{i,t+1}^n - \eta_{i,t}^n) \\
        &= \frac{f_i}{(S_i+f_i)} d \eta_{i,t+1} \\
        &= \lambda_i d \eta_{i,t+1}
    \end{split}
\end{equation}

With the reformulation of these two shocks, I can now write the final version of the price changes model. Price changes in a period can 
be attributed to either a change in intermediate input prices, a change in final demand, or a change in primary input cost. 
Furthermore, this equation is now stationary and does not rely on the strict temporal dimension to implement a simulation model. 
Hence, I write intermediate input price changes as supplier price changes $\Delta p_{i,s}$ and the output price changes as buyer price 
changes $\Delta p_{i,b}$.
\begin{equation} \label{eq:price_changes}
    \Delta p_{i,b} = \frac{1-\gamma_i}{\gamma_i}  \lambda_i d \eta_{i} + \frac{\alpha_{0i}}{\gamma_i} d \varepsilon_{i} + \sum_{j=1}^N \frac{\alpha_{ji}}{\gamma_i} \Delta p_{i,s} \\
\end{equation}

\subsection{Simulation Procedure}

From equation (\ref{eq:price_changes}), simultaneous draws of initial supplier price changes $\Delta P_s$, and the two exogenous shocks 
(final demand and primary input cost) will yield hypothetical price changes, providing a stationary representation of PPI inflation. The 
volatility of sectoral prices over this set of price changes is the first simulated outcome to consider. While equation 
(\ref{eq:price_changes}) dictates the formal structure for this process, it is convenient to rewrite this equation in matrix form. For 
this purpose, I denote the vectors of price changes for buyers and suppliers $\Delta P_{b}, \Delta P_{s}$, vectors of shocks to final 
demand and input cost $d \mathbb{H}, d \mathbb{E}$, the matrix of technical coefficients $\mathbb{A}$ and the diagonal matrices of 
primary input shares, final demand shares and return-to-scale indicators $\hat{\mathbb{A}}_0, \hat{\Lambda}, \hat{\Gamma}$. This 
equation can be understood as the downstream transmission of shocks to final demand and primary input cost together with an initial 
change of intermediate input prices. These shocks are moderated in their impact by their shares of primary input, final demand, and 
intermediate inputs, as these shares represent exposure to shocks from the respective sources. The Returns-To-Scale indicator will, on 
its own, either amplify (for DRS) or diminish (for IRS) the shock impacts. Such an understanding is consistent with the micro-founded 
perspective of cost-minimizing firms that change their prices corresponding to changes in their marginal costs caused by changes to the 
cost of primary and intermediate inputs and final demand.

\begin{equation} \label{eq:downstream_changes}
    \Delta P_{b} = (\hat{\Gamma}^{-1} - I) \hat{\Lambda} d \mathbb{H} + \hat{\Gamma}^{-1} \hat{\mathbb{A}}_0 d \mathbb{E} + \Delta P_{s} \mathbb{A} \hat{\Gamma}^{-1}
\end{equation}

In order to obtain an equation that describes the transmission of price changes and shocks in an upstream direction, similar as in equation 
(\ref{eq:downstream_changes}) does for downstream transmission, one could simply invert the equation to isolate $\Delta P_s$. This 
equation, however, contains the matrix $A^{-1}$, which does not have a sensible interpretation for input-output models, but it also tends 
to hold values that lead to unrealistically large price responses on the given shock input. Hence, I decided against simply choosing 
shock variance values that would return price changes of reasonable size and instead opted to impose an ad-hoc modification of the 
equation, using the matrix of output coefficients $\mathbb{B}$, which is defined as $\mathbb{B} = \hat{X}^{-1} Z$ (in parallel to 
$\mathbb{A}$ being defined as $\mathbb{A} = Z \hat{X}^{-1}$).  Although the equation is not micro-founded as a result of firm 
cost minimization, it can be interpreted similarly to the equation for downstream transmission. The upstream pass-through of price 
pressures can be understood in two steps, where first, the initial change of output prices is modified by the shocks to final demand or 
primary input costs. Since pass-through must be complete in this equation, the remaining change of prices will be transmitted through 
the matrix $\mathbb{B}$, which indicates how much an industry relies on each of its buyers in terms of sales. If all or a sales-weighted 
majority of buyers report a net negative price change (after shocks to final demand and primary input cost), then the industry must 
accommodate this price movement.

\begin{equation} \label{eq:upstream_changes}
    \Delta P_{s} = \hat{\Gamma}^{-1} \mathbb{B} [\Delta P_{b} - (\hat{\Gamma}^{-1} - I) \hat{\Lambda} d \mathbb{H} - \hat{\Gamma}^{-1} \hat{\mathbb{A}}_0 d \mathbb{E}]
\end{equation}

Both equations on the transmission of price changes presented just now describe this process for a singular stage of the production chain.
However, since the model presented here has a round-about structure of intermediate input sourcing, it is possible to consider price 
changes N stages down or up from an initial price draw by applying the function to itself N times over. In doing so, the simulation tries
to disentangle the propagation of shocks along the time dimension, where exogenous shocks would change as the propagation occurs, from the
propagation along the value chain dimension, where exogenous shocks are still present but the price of inputs or outputs has already been 
affected. Consequently, shocks will accumulate as they are re-introduced with every new stage, making this kind of simulation only a 
realistic scenario for a small number of stages since firms would otherwise try to adapt their input mix to balance out the price change. 
Such a situation of restricted adaptation is exactly to consider under very rapid shock occurrence scenarios or with disrupted supply 
chains (see the paper by \textcite{labelle2022GlobalSupplyChain}). Furthermore, with this explicit consideration of the transmission of 
shocks over several stages of the production chains, it seems sensible to introduce the common assumptions that (1) cost shocks are 
transmitted downstream, and (2) demand shocks are transmitted upstream.\footnote{These assumptions are supported by theoretical and 
empirical findings from \textcite{acemoglu2016NetworksMacroeconomyEmpirical}} Such a restriction is sensible for the dimensionality of 
the simulation parameters to consider. Furthermore, it is reasonable not to impose both shocks at all stages of propagation since each 
stage represents an upstream or downstream movement on the production chain, implying a move away from the incidence of final demand 
and primary input cost, respectively. The concrete equations to compute this propagated price changes over several downstream and 
upstream stages are as follows:

\begin{equation} \label{eq:propagation}
\begin{split}
    \Delta P_b^n &= \Delta P_s (\mathbb{A} \hat{\Gamma}^{-1})^n  + \sum_{k=0}^{n-1} (\mathbb{A} \hat{\Gamma}^{-1})^k \hat{\mathbb{A}}_0 d \mathbb{E} \\
    \Delta P_s^n &= (\hat{\Gamma}^{-1} \mathbb{B})^n \Delta P_b - \sum_{k=0}^{n-1} (\hat{\Gamma}^{-1} \mathbb{B})^k (\hat{\Gamma}^{-1} - I) \hat{\Lambda} d \mathbb{H}
\end{split}
\end{equation}

With the equations and intuition for the simulation outlined as such, I can distinguish the inputs required to conduct an investigation
with them into structural ones (e.g. production structure, shares of final demand, and primary input) and variable ones (e.g. shock 
variance, variance of initial price change, Returns-To-Scale). I will vary the latter systematically to explore the sensitivity of 
results to these input choices, while I will choose the Input-Output structures deliberately to draw out certain behaviors through the 
simulation.

\subsection{1-Country, 2-Sector Setup}

%Intent of simple model setup (or two)

A  1-Country, 2-Sector economy has all the essential features to begin the exploration of the relationship between shock exposures 
implied by upstreamness and downstreamness and how that drives differences in price volatility. Even this simplified production structure 
can be used to model ``round-about'' Input-Output linkages while also showing differences in the shares of final demand and primary input 
use relative to total output. Naturally, such a setup is unrealistic, but it is sufficient to draw out the implications of sectoral 
interdependences and differences in shock exposure for price volatility. Any comparison of production chain position and price volatility
will have to be ordinal in this context since, with such a simple framework, ratios of these measures hold little value for interpretation.
With this in mind, I decided on two intentionally coarse IO matrices, where the first matrix (\ref{fig:IO1x2_lowopen}) has its values for 
final demand shares and primary input shares ranked the same over sectors, while the second matrix (\ref{fig:IO1x2_highopen}) has ranked 
them in an inverted way. For both matrices, the columns titled $X$ and $F$ represent total output and final demand, while the row with the
title $VA$ holds values for value added, previously written as $\omega_i$. Finally, the matrix with row and column names (s1,s2) has the 
values for purchases and sales of intermediate inputs, previously written as $Z$. Since I wanted to keep both sectors at the same size 
of output, this setup of IO linkages has led to matrix (\ref{fig:IO1x2_lowopen}) having a dominant diagonal (for rows and columns), 
while the off-diagonal is dominant for matrix (\ref{fig:IO1x2_highopen}) (for rows and columns). In terms of generalization for larger 
IO matrices, this implies that producers in the system represented by matrix (\ref{fig:IO1x2_lowopen}) rely majorly on ``domestic'' 
inputs, while producers in the system for matrix (\ref{fig:IO1x2_highopen}) do so with ``foreign'' inputs. I term these IO linkage 
scenarios ``low openness'' and ``high openness'' respectively.

\begin{figure}[H]
    \makebox[\textwidth][c]{%
    \subfloat[IO matrix for 1-Country, 2-Sector setup with low openness]{\scalebox{0.8}{\includegraphics{pictures/IO_table_1x2_lowopen.png}}
    \label{fig:IO1x2_lowopen}}
    \quad
    \subfloat[IO matrix for  1-Country, 2-Sector setup with high openness]{
        \scalebox{0.8}{\includegraphics{pictures/IO_table_1x2_highopen.png}}
        \label{fig:IO1x2_highopen}}
    }
\end{figure}

From the IO matrix (\ref{fig:IO1x2_lowopen}), it is apparent that sector $s1$ has a final demand share of 0.2 and a primary input share of 0.2.
Conversely, sector $s2$ has a final demand share of 0.4 and a primary input share of 0.4. This IO structure results in upstreamness values of
$s1: 4.286,\hspace{3mm} s2: 2.857$ and downstreamness values of $s1: 4.286,\hspace{3mm} s2: 2.857$. 
Evidently, with the symmetric structure of this IO matrix, the upstreamness and downstreamness values here are the same. Further, this structure of 
low openness (dominant matrix diagonal) has also created a positive ``correlation'', between upstreamness and downstreamness.
From the IO matrix (\ref{fig:IO1x2_highopen}), it is apparent that sector $s1$ has a final demand share of 0.3 and a primary input share of 0.5.
Conversely, sector $s2$ has a final demand share of 0.5 and a primary input share of 0.3. This IO structure results in upstreamness values of
$s1: 3.478,\hspace{3mm} s2: 3.043$ and downstreamness values of $s1: 3.043,\hspace{3mm} s2: 3.478$.
Evidently, with the symmetric structure of this IO matrix, the upstreamness and downstreamness values here are the same. Further, this structure of 
high openness (dominant matrix diagonal) has also created a negative ``correlation'', between upstreamness and downstreamness.

\subsection{2-Country, 2-Sector Setup}

Given my theoretical framework, a 1-Country, 2-Sector economy already has all the essential features to explore the basic relationship 
between production chain position and price volatility. Based on that, the 2-Country, 2-Sector setup presents a first step toward a more 
complete representation of a globally integrated production system. This applies to several specific features, specifically increased 
diversification of sourcing and sales across the various country-sectors, but also shocks to final demand and primary input cost that 
are correlated by sector or country next to basic country-sector shocks (regimes of mixed origin shocks are also possible). Such an IO 
matrix with two countries and two sectors each is not a realistic representation of a globally integrated production system, but it does 
strike a reasonable balance between the tractability of a simple model and the dangers of building a ``black box'' model in trying to 
gain external validity. This has been critiqued by \textcite{caliendo2015EstimatesTradeWelfare} about previous CGE-based approaches, 
and a full extension of my framework to the full extent of available data would incur the same problem. Finally, the assembly of four 
country-sectors with this setup will be instrumental in ensuring that any conclusions about rankings of volatility do not arrive because 
the sector with lower price volatility has a certain position on the production chain coincidentally. With these things in mind, I construe
two IO matrices where, again, for the first matrix (\ref{fig:IO2x2_lowopen}) the rankings of final demand share and value added share 
align, and where for the second matrix (\ref{fig:IO2x2_highopen}) these rankings are inverted. Overall, I have decided to introduce 
the shares for primary inputs and final demand so that the sector-wise comparison is clear, as is the country-wise comparison. 
However, a general structure in each country remains, with one sector more dependent on final demand or primary inputs respectively.
The notations of the matrices depicted 
are the same as for the 1-Country, 2-Sector with output $X$, final demand $F$, value added $VA$ and intermediate inputs $Z$. The only 
difference is that the country-sector denominations now include a country indexation (A\_s1,A\_s2,B\_s1,B\_s2). With all country-sectors 
of the same size in terms of output, this creates scenarios with matrix in figure (\ref{fig:IO2x2_lowopen}) having a dominant block-diagonal,
and the matrix in figure (\ref{fig:IO2x2_highopen}) having a dominant off-diagonal blockwise. Similarly to before, a dominant block-diagonal
for rows and columns implies that country-sectors rely on input sourcing and output sales on domestic partners at a majority (weakly).

\begin{figure}[H]
    \centering 
    \includegraphics[width=7.5cm]{pictures/IO_table_2x2_lowopen.png}
    \caption{\label{fig:IO2x2_lowopen} IO matrix for 2-Country, 2-Sector setup with low openness}
    
\end{figure}

\begin{figure}[H]
    \centering
    
    \includegraphics[width=7.5cm]{pictures/IO_table_2x2_highopen.png}
    \caption{\label{fig:IO2x2_highopen} IO matrix for 2-Country, 2-Sector setup with high openness}
\end{figure}

From the IO matrix (\ref{fig:IO2x2_lowopen}), it is apparent that sector $A\_s1$ has a final demand share of 0.4 and a primary input share of 0.4, and 
sector $A\_s2$ has a final demand share of 0.6 and a primary input share of 0.6. Conversely, sector $B\_s1$ has a final demand share of 0.3 and a 
primary input share of 0.3, and sector $B\_s2$ has a final demand share of 0.5 and a primary input share of 0.5. This IO structure results in 
upstreamness values of
$A\_s1: 2.402,\hspace{3mm} A\_s2: 1.924,\hspace{3mm} B\_s1: 2.616,\hspace{3mm} B\_s2: 2.2$ and downstreamness values of 
$A\_s1: 2.357,\hspace{3mm} A\_s2: 1.921,\hspace{3mm} B\_s1: 2.636,\hspace{3mm} B\_s2: 2.228$. As for the 1-country, 2-sector setup, this structure of low openness (dominant matrix diagonal) has 
created a positive ``correlation'', between upstreamness and downstreamness.
From the IO matrix (\ref{fig:IO2x2_highopen}), it is apparent that sector $A\_s1$ has a final demand share of 0.4 and a primary input share of 0.4, and 
sector $A\_s2$ has a final demand share of 0.6 and a primary input share of 0.6. Conversely, sector $B\_s1$ has a final demand share of 0.3 and a 
primary input share of 0.3, and sector $B\_s2$ has a final demand share of 0.5 and a primary input share of 0.5. This IO structure results in 
upstreamness values of 
$A\_s1: 1.853,\hspace{3mm} A\_s2: 2.264,\hspace{3mm} B\_s1: 2.026,\hspace{3mm} B\_s2: 2.503$ and downstreamness values of 
$A\_s1: 2.481,\hspace{3mm} A\_s2: 2.01,\hspace{3mm} B\_s1: 2.305,\hspace{3mm} B\_s2: 1.85$.  As for the 1-country, 2-sector setup, this structure of 
high openness (dominant matrix diagonal) has created a negative ``correlation'', between upstreamness and downstreamness. based on the evidence provided by 
\textcite{antras2018MeasurementUpstreamnessDownstreamness}, the correlation between upstreamness and downstreamness should increase as openness of the 
international production system increases. Hence, it should be considered critical that these IO tables, set up for a multi-country simulation, do not show 
this pattern. I will discuss the reasons for this pattern and how a future extension of this model might be able to resolve this inconsistency.


 %Basic model framework, 1x2 frame, 31x17 frame
\section{Simulation Results}
\label{sec:simres}

In this section, I present and discuss the simulation results for the 1-country, 2-sector, and 2-country, 2-sector models, 
respectively, with downstream transmission of cost shocks and upstream transmission of demand shocks. Each shock scenario will be 
presented through two graph panels of 16 plots each. Each panel of plots has the results from different shock variance parameters on 
the columns and results from different variance values for the initial price changes on the rows. The first graph for each pair 
presents results on the first transmission stage but across several Returns-To-Scale (RTS) indicator values. The second graph for each 
pair shows the propagation of shocks to price volatility across up to five stages but only for one combination of RTS.

\subsection{2-Sector Economy}

Graph (\ref{fig:1x2cost_y1}a) provides the first insights into the one-stage downstream transmission of primary input cost shocks on 
price volatility in equation (\ref{eq:downstream_changes}) for the IO matrix shown in figure (\ref{fig:IO1x2_lowopen}). In this scenario, a more 
downstream sector (sector indicators with downstreamness in legend) will be less volatile (volatilities on y-axis) if the cost shock 
distribution dominates the distribution of initial price changes in variance (upper triangular of plot panel). Generally, the simulated 
volatilities for all sectors and shock scenarios are smaller as the TTS indicator increases (RTS indicators on x-axis). Considering the 
role of the RTS indicator matrix $\Gamma$ in equation (\ref{eq:downstream_changes}), such an effect is mathematically reasonable. 
The theoretical interpretation of this observation can be seen as partial equilibrium adaption with economies of scale in this basic 
model framework. The model intuition suggests that a positive (negative) primary input cost shock or initial price change leads to an 
increase (decrease) in the marginal cost and, hence, buyer price. Since this simulation is conducted under an assumption of perfect 
competition, the price change must be due to a change in the supply produced supply. In this sense, the theoretical role of RTS can 
be understood in terms of the production structure enacting an amplifying (case of decreasing RTS, $\gamma_i < 1$) or diminishing 
(case of increasing RTS, $\gamma_i > 1$) effect on this change. If, as for these simulations, all sectors have the same RTS values, 
it is reasonable to expect decreasing volatilities as RTS values increase. 

Graph (\ref{fig:1x2cost_y1}b) displays the multi-stage downstream propagation in equation (\ref{eq:propagation}) of primary input cost shocks, 
specifically for the case of constant RTS (differences for other RTS are small).\footnote{The volatility values in graph (\ref{fig:1x2cost_y1}a) 
for the RTS indicator equal to one, hence are the same in each panel as the values for stage one volatilities in graph (\ref{fig:1x2cost_y1}b).} 
This graph shows that the pattern of the more downstream sectors being less volatile (volatilities on y-axis) in the cross-section of sectors 
(sector indicators with downstreamness in legend) persists even as the simulation propagates the economy downstream in total (stages 
on the x-axis). Again, as for the one-stage transmission in the previous paragraph, this observation holds only for shock scenarios
where the variance of the cost shock distribution is larger than the variance of the initial price change 
distribution (upper triangular of plot panel). Volatility will also accumulate with downstream propagation stages in the shock scenarios of this 
upper triangular, while it diffuses for the cases in the lower triangular. Furthermore, it is apparent that for the given level 
of RTS, in the lower triangular of graph (\ref{fig:1x2cost_y1}b), the more downstream sector will become less volatile faster and 
even switch the ranking as with propagation into more downstream stages. Plots of the same kind as (\ref{fig:1x2cost_y1}b) but 
for different levels of RTS show that the propagation dynamic will be more pronounced the higher RTS are. The appendix \ref{sec:1x2_sim}
contains the same graphs shown here but for the simulation conducted on the IO matrix in figure (\ref{fig:IO1x2_highopen}). As previously
mentioned, the supposed main difference between the two was the reversal of the ranking of value added shares, implying that this
new IO matrix deviates from the positive correlation between upstreamness and downstreamness shown in the data. Remarkably, both graphs 
do not show any major qualitative differences in the pattern targeted for this project (mind the color change for the more downstream 
sector). 

\begin{figure}[H]
    \includegraphics[width=\textwidth]{graphs/1x2_scenarioplot_cost_shocks_1.0-1.0_year1.png}
    \caption{\label{fig:1x2cost_y1} Cost shock propagation}
\end{figure}

Moving on to final demand shocks, graph (\ref{fig:1x2demand_y1}a) provides the first insights to examine the one-stage upstream transmission
of final demand shocks in equation (\ref{eq:upstream_changes}), again for data simulated on the IO matrix in figure (\ref{fig:IO1x2_lowopen}).
This graph shows that at the first stage of upstream demand shock transmission, more upstream sectors (sector indicators with upstreamness in 
legend) are more volatile (volatilities on y-axis) across all values of RTS (RTS indicators on x-axis), and all combinations of variances for 
final demand shocks and initial price changes. Since there is little difference between the plots in each row but a significant difference 
between those in each column, the variance of demand shocks is of low relative importance compared to the variance of initial price changes.
A higher variance of initial price changes intuitively results in a higher price volatility. For no variance in initial price changes, 
an expected pattern is visible for Constant RTS, where the demand shock has no impact, no matter its variance. Generally and consistent
with the pattern observed for cost shocks before, volatility decreases with RTS across all sectors and in all stage one shock 
scenarios. 

Graph (\ref{fig:1x2demand_y1}b) displays the multi-stage upstream propagation in equation (\ref{eq:propagation}) of final demand shocks, 
specifically for the case of decreasing RTS (differences for other RTS are small).\footnote{The volatility values in graph (\ref{fig:1x2demand_y1}a) 
for the RTS indicator equal to 0.8, hence are the same in each panel as the values for stage one volatilities in graph (\ref{fig:1x2demand_y1}b).}
As already observed for the first stage transmission in graph (\ref{fig:1x2demand_y1}a), more upstream sectors in the cross-section (sector indicators 
with upstreamness in legend) continue to be more volatile (volatilities on y-axis) in all stages (stages on the x-axis) of shock propagation and with 
no qualifications. Furthermore, volatility will only accumulate (or persist) with upstream propagation (see rows 1-2 in graph 
(\ref{fig:1x2demand_y1}b)) if the variance of the initial price change distribution is sufficiently low. This pattern is even more pronounced 
for higher levels of RTS. The appendix \ref{sec:1x2_sim} contains the same graphs as shown here, but for the simulation conducted on the IO matrix in figure 
(\ref{fig:IO1x2_highopen}). Again, the differences between these results are mostly quantitative and do not affect the patterns reported.

\begin{figure}[H]
    \includegraphics[width=\textwidth]{graphs/1x2_scenarioplot_demand_shocks_0.8-0.8_year1.png}
    \caption{\label{fig:1x2demand_y1} Demand shock propagation}
\end{figure}

% Concrete discussion of mechanism
Overall, the patterns observed from the simulated data appear consistent with what has been shown from the empirical data. Sectors with 
higher upstreamness are more volatile in their prices, and sectors with higher downstreamness are less volatile. No major difference 
exists between the simulated results for the IO matrix with ``high openness'' and the one with ``low openness''. This implies that 
the mechanism behind the results for the 1-country, 2-sector setup is all about the sectoral shares of final demand and primary inputs. 
The sector with the higher share of final demand will have lower upstreamness and, hence, lower price volatility. Conversely, the sector
with a higher share of primary inputs will have lower downstreamness and, hence, higher price volatility. This emphasizes the importance of
understanding how the direct shock exposure reflected in these shares gives rise to both the production chain positions and the
price volatility. As an indicator of direct shock exposures, the shares determine the exposures to indirect shocks and, hence, the 
aggregate relations observable in the data. This model grossly oversimplifies the situation and most likely overemphasizes 
the direct shares' role and the impact of purely country-sector shocks in determining price volatility. Since such a simplification 
is necessary to explain the underlying mechanisms but the dangers of losing external validity, the next section is a step toward 
qualifying this simplification. 


\subsection{2-Country, 2-Sector Economy}

The first stage downstream transmission of cost shocks seen in graph (\ref{fig:2x2cost_i_y1}a) provides insights on how shocks to primary input 
costs drive differences in price volatility with two countries and two sectors. The simulations shown here are conducted for the IO 
matrix in figure (\ref{fig:IO2x2_lowopen}). As before, the x-axis shows RTS values, the y-axis volatilities, and the legend country-sector
indicators with downstreamness values. The basic result is consistent with the one from the 1-country,2-sector 
setup in the previous section, and shows that more downstream sectors are less volatile in prices. The differences in simulated price 
volatility are more pronounced for higher variances of cost shock. This overall pattern is reversed only if cost shock variance is very 
low compared to the variance of initial price changes, which is in contrast to the 1x2 IO matrix simulation, where reversal was observed 
for equal variance values. This is most likely a result of the higher share of value added in this 2x2 IO matrix compared to the 1x2 
IO matrix. Otherwise, it appears to be the case still that higher RTS values cause lower simulated volatilities and that 
higher variances of initial price change cause higher simulated volatilities at this first stage of shock transmission. 

Graph (\ref{fig:2x2cost_i_y1}b) displays the multi-stage downstream propagation in equation (\ref{eq:propagation}) of primary input cost shocks, 
specifically for the case of constant RTS (differences for other RTS are small).\footnote{The volatility values in graph (\ref{fig:2x2cost_i_y1}a) 
for the RTS indicator equal to 1.0, hence are the same in each panel as the values for stage one volatilities in graph (\ref{fig:2x2cost_i_y1}b).}
The overall pattern 
between volatility and downstreamness in the country-sector cross-section persists as the shock propagates downstream along the value 
chain. Moreover, volatility accumulates with propagation down the chain for higher variance values (graph (\ref{fig:2x2cost_i_y1}b), 
columns 3 and 4) but is diffused otherwise (columns 1 and 2). These results are also found again for simulations of the second IO matrix 
with country-sector shocks, visualized in the appendix \ref{sec:2x2_sim}. Appendix \ref{sec:2x2_sim} also contains the graph for country-sector
shocks to primary input cost for the simulations on IO matrix (\ref{fig:IO2x2_highopen}), the graphs for other shocks were omitted since they 
would not have added much to this paper. Furthermore, I find that cost shocks 
originating purely at the country or sector level do not change these simulated patterns significantly. However, they do induce a 
convergence of volatilities for the multi-stage shock propagation, and they weaken the dominance of cost shocks over initial price 
changes in terms of impact on volatility (as observed in graph (\ref{fig:2x2cost_i_y1}a)). Finally, the simulations under shock scenarios 
that mix country with country-sector shocks and sector with country-sector shocks show some more complicated patterns.

\begin{figure}[H]
    \includegraphics[width=\textwidth]{graphs/2x2_scenarioplot_i_cost_shocks_RS0.8_year1.png}
    \caption{\label{fig:2x2cost_i_y1} Country-sector cost shock propagation}
\end{figure}

Moving on to final demand shocks, graph (\ref{fig:2x2demand_i_y1}a) provides insights into how shocks to final demand drive differences in 
price volatility in the first stage of upstream transmission. The simulations shown here are conducted for the IO 
matrix in figure (\ref{fig:IO2x2_lowopen}). As before, the x-axis shows RTS values, the y-axis volatilities, and the legend country-sector
indicators with upstreamness values. More upstream sectors are also more volatile across all values 
for RTS and variance of shocks and initial price change. In general, volatility decreases as RTS increase, but 
it also appears to be the case that the initial price changes are more impactful for volatilities than the cost shock. This is 
evident from the increase in the level of volatility with the rows in graph (\ref{fig:2x2demand_i_y1}a), as well as the minor differences 
only between the graph's columns. Finally, the patterns for Constant RTS that are visible for zero variance of initial 
price changes do not reappear for the other plot rows. All of this is very consistent with the observed patterns from results for the 
1-country, 2-sector setup.

Graph (\ref{fig:2x2demand_i_y1}b) displays the multi-stage upstream propagation in equation (\ref{eq:propagation}) of final demand shocks, 
specifically for the case of decreasing RTS (differences for other RTS are small).\footnote{The volatility values in graph (\ref{fig:2x2demand_i_y1}a) 
for the RTS indicator equal to 0.8, hence are the same in each panel as the values for stage one volatilities in graph (\ref{fig:2x2demand_i_y1}b).}
The general relation of upstreamness and price volatility in the 
country-sector cross-section discussed just now also persists for the propagation up the value chain as shown in graph (\ref{fig:2x2demand_i_y1}b). 
Again, the dominance of the initial price changes is apparent here since volatility diffuses as the shocks propagate up the chain. This 
diffusion of volatility occurs for plot rows with non-zero price change variances (graph (\ref{fig:2x2demand_i_y1}b), rows 2-4), and 
accumulation only happens if initial price changes have zero variance (row 1). These results are also found again with one exception 
for simulations of the second IO matrix (\ref{fig:IO2x2_highopen}) with country-sector shocks, visualized in the appendix \ref{sec:2x2_sim}. 
The graphs for other shocks to the setup of IO matrix (\ref{fig:IO2x2_highopen}) were omitted since they would not have added much to this paper.
This exception is that at the first stage, the two most upstream sectors do not have a significant difference in volatility despite 
having a difference in upstreamness. However, as the shock propagates up the value chain, this is corrected to be consistent with the 
expected pattern. Comparing the 2-Country, 2-Sector setup, and the 1-Country, 2-Sector setup, results on the impact from country-sector shocks to 
final demand are highly similar. Even the introduction of correlated shocks at the country and sector level or with combined shock 
origins (country and country-sector, sector and country-sector) do not bring significant changes in the simulated patterns. Most likely, 
this is related to the generally weak impact of final demand shocks relative to the initial price change in this framework.


\begin{figure}[H]
    \includegraphics[width=\textwidth]{graphs/2x2_scenarioplot_i_demand_shocks_RS0.8_year1.png}
    \caption{\label{fig:2x2demand_i_y1} Country-sector demand shock propagation}
\end{figure}

% Concrete discussion of mechanism is contrast to 2-sector example

Overall, the patterns described for the simulations on the 2-Country, 2-Sector setup are still consistent with the stylized facts 
found in the data. The introduction of correlated shocks at the country or sector level as stand-alone scenarios does not 
cause a major break in the pattern. Correlated shocks at the country or sector level appear to drive a ``comovement'' of prices that 
emerges over a decrease in the difference between country-sector volatilities for the muti-stage propagation plots. This is a sensible pattern since 
shocks that affect a set of country-sectors equally should cause a convergence by themselves already. But moreover, this convergence should be 
accelerated through the IO linkages between country-sector during propagation up or down the value chain. Furthermore, there is no significant difference 
to observe in qualitative terms between the IO matrix setup for high and low openness. The continued stability of the pattern, as empirically observed 
in the data, is a sign of strong internal consistency of 
the model behind these patterns. The shares of final demand and primary input against total output appear to be the main drivers 
since they determine the direct shock exposure and the total (indirect) exposure (upstreamness, downstreamness) through the 
network as well. As such, the model equations (\ref{eq:propagation}) provide a formal framework of how shocks to final demand (primary input cost)
affect country-sectors in a production system directly and indirectly through the network. This is visible already for the first-stage transmission 
but becomes even more apparent as shocks are simulated to propagate up (down) the production chain over multiple stages. However, at this point, 
the lack of an observable difference in results between the IO matrices with high and low ``openness''
raises the question if this framework might be over relying on the mechanism around final demand shares and primary input shares. 
This might be an interesting challenge for a future extension of the model, as discussed in section \ref{sec:limit}.

%From the overall discussion of the results for the 2-Country, 2-Sector setup and the comparison to the 2-Sector setup, the patterns are 
%remarkably consistent. This indicates that the model outlined in this paper has an internally consistent mechanism that is robust to 
%changes in its inputs . 
 %different sets of 1x2 (maybe 2x2) and the 31x17 for external validity
\section{Limitations and Extensions}
\label{sec:limit}

The results presented in the previous section show that under the presented settings, the model is able to reproduce both the 
positive correlation between upstreamness and price volatility and the negative correlation between downstreamness and price volatility.
This should be seen as a success of the model since the model-based reproduction of these stylized facts was the primary goal 
of this paper. However, the model is still grossly oversimplified as a representation of a globally integrated production system and uses 
very coarse restrictions on implicit and explicit parameters. This section will outline some of the shortcomings that are visible already 
now, how the model might be modified to incorporate solutions to these shortcomings, and what the implications of such modifications would 
be. 

%% Oversight in setup change in openness implies wrong correlations between prod chain measures.
First, in setting up the IO matrices used for the simulations, I intentionally made it so that one matrix would have a negative correlation 
between upstreamness and downstreamness (or rather inverted rankings of these measures) and one would have a positive correlation between
these variables (parallel ranking). I also wanted to keep (country-)sectors at equal output size for the whole table in order to 
exclude the factor of size differences as a determinant of differences volatility and keep the range of parameters to explore at a 
sensible size. The resulting IO matrices had the correlations as intended, but the one that showed a ``high openness'' has a negative 
correlation between upstreamness and downstreamness, making this feature explicitly stand in contrast to the data. While this is 
a problematic feature of the setup, I decided it was the preferable option compared to opening the project up to a discussion 
about sector size since that would have extended into issues about granularity. Including the issue of granular sectoral economies would 
have exceeded the scope of this thesis in terms of model formulation and the availability of disaggregated data. However, this is 
certainly an extension to the project that is worthwhile to consider for the future.


%% Issues with structural parameters
Second, the model presented in section \ref{sec:model} has no inter-industry barriers or frictions. Most notably, 
such elements include trade costs and pass-through. Regarding trade costs, such a modification would be easily achievable, e.g. in the 
form of standard ``iceberg'' trade costs as a square matrix of constants with values larger or equal one to represent either the price increase 
for any inter-industry purchase. This addition might be instrumental in making simulation scenarios with multiple countries more 
realistic. The chosen values should be systematically varied to explore the sensitivity of results to different trade costs or 
undergo empirical justification (for a large-scale quantitative model implementation). Regarding pass-through, there is a breadth of empirical 
studies to show that pass-through is often imperfect but may also be heterogeneous across sectors (or firms) and dependent on the state of the 
economy (e.g. inflation rate). A simple modification to incorporate pass-through as a vector of constants applied to the marginal cost 
equation would not be sufficient since these constants would cancel each other out when reformulating the model into price changes.
However, a more serious attempt at including heterogeneous pass-through would have to decide whether to include such a feature as 
price rigidity or variable mark-ups and hence rely on models of price setting or imperfect competition to supplement the current framework.
Such an extension would also make the treatment of non-constant Returns-To-Scale values, as discussed in section \ref{sec:model}, significantly
more coherent. The inclusion of models of price setting into the model framework would have the additional advantage of modeling a realistic
distribution of price changes, especially in terms of frequency e.g. via menu costs models. The current framework has no such inhibition,
which should imply price changes that are too frequent and, hence, too high volatility. However, such a model would necessarily 
have to include a dynamic version of the optimization problem since fixing prices over time requires such a choice to be dynamically optimal 
(at least in expectation).

Third, the model framework I use to generate my results relies on a strong assumption of its Input-Output structure being exogenous.
However, firms may account for the expected price volatility of their supplier when making a sourcing decision.
Such an endogeneity in the data-generating process behind my model framework would not make the observation of the effects of
small, idiosyncratic shocks on price volatility infeasible in and of itself. However, it does introduce another potential explanation 
for the observed relations between production chain position and price volatility. With the formation of IO links being endogenous on 
the expected price volatility, it could be that sectors with naturally high price volatility are placed further upstream as a result 
of this process.

%% Issues with shock specification
Fourth, it should be an aspect of some criticism that the simulations are based on relatively arbitrary shock distributions. The choice 
of a normal distribution with mean zero for shocks to final demand, primary input cost, and initial price changes is not a problem as 
such since it is a convenient and tractable way to introduce variation of stochastic components to the model. The first moment of the 
empirically observed cross-sectional price change distribution is slightly larger than zero since inflation is the usual trend (rather 
than deflation). Hence, while a first moment of zero is uncritical for the shock distributions, it evidently is inconsistent with 
empirical facts on price change distributions. However, this paper aims to explain simple patterns observed in the data through a basic 
model framework and not match volatilities quantitatively; such inconsistency is still acceptable. The justification for the variances 
of these distributions is to be seen more critically since a key part of the simulation result is a visual judgment of the relative 
importance of shocks and initial price changes. This limitation could be fixed somewhat through a more fine-grained exploration of 
different variance values and trying to find the parameter combinations that correspond to decent fits with empirically observed price 
volatilities. A more rigorous attempt would involve a full model calibration of the distribution parameters to match empirical stylized 
facts, e.g. moments of the price change distributions.

%% Issues with data and external validity
%Fifth, lack of data on prices for service sectors restricts the analysis in this paper significantly. Obviously, this is a common 
%problem for studies with sectoral disaggregation since data on sectors is much more difficult to survey and harmonise across countries. 
%However, the exclusion of service sectors in the case of this model systematically increases the share of
%Does not include data on services - pushes them into the value added component, which is a gross simplification since services
%might still be traded, especially within a country




 %stickiness, non-roundabout, network formation, aggregation
%\section{Policy Implications} %
\section{Conclusion}
\label{sec:conc}

Throughout the last sections, I have outlined novel empirical facts on the relation between a sector's position on the value chain 
and its price volatility. Concretely, I show that higher upstreamness, representing a stronger exposure to the spillovers of shocks 
to final demand, is associated with higher price volatility. Similarly, I also provide evidence that higher downstreamness,
implying a stronger exposure to the spillovers of shocks to primary input cost, is associated with lower price volatility. I attempt 
to provide some theoretical intuition for the mechanisms behind this fact by constructing a model of N-sector cost minimization
with ``round-about'' Input-Output linkages. I conduct simulations for the upstream propagation of final demand shocks and downstream 
propagation of primary inputs cost shocks for both a 2-Sector and a 2-Country, 2-Sector setup and show that the model simulations 
broadly reproduce the empirical pattern at the first stage of shock transmission. Higher upstreamness is associated with higher 
price volatility, while higher downstreamness is associated with lower price volatility. This pattern is robust over different values for
Returns-To-Scale and simple representations of low and high openness for the Input-Output system. The simulations are run with a short-term 
time in mind. Thus, results should not be considered valid for longer durations since this would give firms the opportunity to adapt 
their production linkages, violating a fundamental assumption of exogenous production structure.

In an attempt to simulate the propagation of shocks along the value chain, separately from propagation over time, I show that in my 
simulated model, the above relations hold in the sectoral cross-section as shock transmission is simulated several stages up 
(for demand shocks) or down (for cost shocks) the production chain. Further theoretical implications show that correlated shocks at 
the country or sector level do not break these general patterns but induce sectoral co-movement as they decrease the differences in 
volatility for a simulated propagation of the shock beyond a first-stage transmission. With the results outlined just now, the model 
shows a strong internal consistency, but it also has some obvious shortcomings, as discussed in section \ref{sec:limit}. However, many of 
these shortcomings have been previously encountered and documented in various contributions to the literature. Hence, a future extension 
of this project with a more complete formal model has the potential to remedy these shortcomings.

Despite the apparent shortcomings, this paper is a valuable contribution to the literature on sectoral price dynamics with 
production linkages. I document two new stylized facts for international monetary economics with production linkages. Furthermore, I 
contribute to understanding upstreamness and downstreamness as measures of shock exposure, a perspective that is underrepresented 
in international macroeconomics. Finally, I provide a simple framework to model how the production linkages that constitute these measures
also drive the propagation of shocks and, hence, give rise to differences in price volatility.

\newpage
\appendix    
\section{Appendix}

\subsection{Additional Empirical Visualisations}
\label{sec:add_viz}

\begin{figure}[H]
    \includegraphics[width=\textwidth]{graphs/volat_year_dist.png}
    \caption{\label{fig:year_violin} Violin plots of volatilities within each year}
\end{figure}

\begin{figure}[H]
    \includegraphics[width=\textwidth]{graphs/volat_sector_dist.png}
    \caption{\label{fig:sector_violin} Violin plots of volatilities within each sector}
\end{figure}

\begin{figure}[H]
    \includegraphics[width=\textwidth]{graphs/volat_country_dist.png}
    \caption{\label{fig:country_violin} Violin plots of volatilities within each country}
\end{figure}

\subsection{Additional Regression Tables}
\label{sec:add_reg}

\input{tables/reg_ppi_volat_std_tot_bwd.tex}

\input{tables/reg_ppi_volat_std_tot_fwd.tex}

\input{tables/reg_ppi_volat_std_tot_gvc.tex}

\input{tables/reg_ppi_volat_std_year1995-2002.tex}

\input{tables/reg_ppi_volat_std_year2003-2011.tex}


\subsection{Additional Simulation Plots for 2-Sector model}
\label{sec:1x2_sim}

% \begin{figure}[H]
%     \includegraphics[width=\textwidth]{graphs/1x2_scenarioplot_cost_shocks_1.0-1.0_year0.png}
%     \caption{\label{fig:1x2cost_y0} Cost shock propagation with high openness}
% \end{figure}

% \begin{figure}[H]
%     \includegraphics[width=\textwidth]{graphs/1x2_scenarioplot_demand_shocks_0.8-0.8_year0.png}
%     \caption{\label{fig:1x2demand_y0} Demand shock propagation with high openness}
% \end{figure}

\subsection{Additional Simulation Plots for 2-Country, 2-Sector model}
\label{sec:2x2_sim}

% \begin{figure}[H]
%     \includegraphics[width=\textwidth]{graphs/2x2_scenarioplot_s_cost_shocks_RS1.0_year1.png}
%     \caption{\label{fig:2x2cost_s_y1} Sectoral cost shock propagation with low openness}
% \end{figure}

% \begin{figure}[H]
%     \includegraphics[width=\textwidth]{graphs/2x2_scenarioplot_c_cost_shocks_RS1.0_year1.png}
%     \caption{\label{fig:2x2cost_c_y1} Country cost shock propagation with low openness}
% \end{figure}

% \begin{figure}[H]
%     \includegraphics[width=\textwidth]{graphs/2x2_scenarioplot_si_cost_shocks_RS1.0_year1.png}
%     \caption{\label{fig:2x2cost_si_y1} Sectoral and country-sectoral cost shock propagation with low openness}
% \end{figure}

% \begin{figure}[H]
%     \includegraphics[width=\textwidth]{graphs/2x2_scenarioplot_ci_cost_shocks_RS1.0_year1.png}
%     \caption{\label{fig:2x2cost_ci_y1} Country and country-sectoral cost shock propagation with low openness}
% \end{figure}

% \begin{figure}[H]
%     \includegraphics[width=\textwidth]{graphs/2x2_scenarioplot_s_demand_shocks_RS0.8_year1.png}
%     \caption{\label{fig:2x2demand_s_y1} Sectoral demand shock propagation with low openness}
% \end{figure}

% \begin{figure}[H]
%     \includegraphics[width=\textwidth]{graphs/2x2_scenarioplot_c_demand_shocks_RS0.8_year1.png}
%     \caption{\label{fig:2x2demand_c_y1} Country demand shock propagation with low openness}
% \end{figure}

% \begin{figure}[H]
%     \includegraphics[width=\textwidth]{graphs/2x2_scenarioplot_si_demand_shocks_RS0.8_year1.png}
%     \caption{\label{fig:2x2demand_si_y1} Sectoral and country-sectoral demand shock propagation with low openness}
% \end{figure}

% \begin{figure}[H]
%     \includegraphics[width=\textwidth]{graphs/2x2_scenarioplot_ci_demand_shocks_RS0.8_year1.png}
%     \caption{\label{fig:2x2demand_ci_y1} Country and country-sectoral demand shock propagation with low openness}
% \end{figure}

% \begin{figure}[H]
%     \includegraphics[width=\textwidth]{graphs/2x2_scenarioplot_s_cost_shocks_RS1.0_year0.png}
%     \caption{\label{fig:2x2cost_i_y0} Country-sectoral cost shock propagation with high openness}
% \end{figure}

% \begin{figure}[H]
%     \includegraphics[width=\textwidth]{graphs/2x2_scenarioplot_i_demand_shocks_RS0.8_year0.png}
%     \caption{\label{fig:2x2demand_i_y0} Country-sectoral demand shock propagation with high openness}
% \end{figure}

\newpage

\newrefcontext[sorting=nyt]
    \printbibliography

\end{document}